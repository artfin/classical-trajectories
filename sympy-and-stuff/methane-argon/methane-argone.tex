\documentclass[12pt]{article}

\usepackage[T1]{fontenc}
\usepackage[utf8]{inputenc}
\usepackage[russian]{babel}

% page margin
\usepackage[top=2cm, bottom=2cm, left=2cm, right=2cm]{geometry}

% AMS packages
\usepackage{amsmath}
\usepackage{amssymb}
\usepackage{amsfonts}
\usepackage{amsthm}

% for strut keyword
\usepackage{mathtools}

% blackboard lettering
\usepackage{dsfont}
\usepackage{bbm}

\usepackage{fancyhdr}
\pagestyle{fancy}

\newcommand{\bbU}{\mathbb{U}}
\newcommand{\bbG}{\mathbb{G}}
\newcommand{\bbE}{\mathbb{E}}
\newcommand{\bbS}{\mathbb{S}}
\newcommand{\bbV}{\mathbb{V}}
\newcommand{\bfj}{\mathbf{j}}
\newcommand{\bfJ}{\mathbf{J}}
\newcommand{\mL}{\mathcal{L}}
\newcommand{\mH}{\mathcal{H}}

\newcommand{\mf}{\mathbf}

\newcommand{\lb}{\left(}
\newcommand{\rb}{\right)}
\newcommand{\lcb}{\left\{}
\newcommand{\rcb}{\right\}}

\begin{document}

\begin{gather}
	\begin{aligned}
			X_1 &= 0 \\
			Y_1 &= 0 \\
			Z_1 &= R \\
	\end{aligned}
	\quad
	\begin{aligned}
			X_2 &= l \sin \Theta \cos \Phi \\
			Y_2 &= l \sin \Theta \sin \Phi \\
			Z_2 &= l \cos \Theta 
	\end{aligned}
	\quad 	
	\begin{aligned}
			X_3 &= \sqrt{2} l \sin \lb \Theta + \frac{\pi}{2} \rb \cos \Phi \\
			Y_3 &= \sqrt{2} l \sin \lb \Theta + \frac{\pi}{2} \rb \sin \Phi \\
			Z_3 &= \sqrt{2} l \cos \lb \Theta + \frac{\pi}{2} \rb 
	\end{aligned}
	\quad
	\begin{aligned}
			X_4 &= \sqrt{2} l \sin \lb \Theta + \frac{\pi}{2} \rb \cos \lb \Phi + \frac{\pi}{2} \rb \\
			Y_4 &= \sqrt{2} l \sin \lb \Theta + \frac{\pi}{2} \rb \sin \lb \Phi + \frac{\pi}{2} \rb \\
		Z_4 &= \sqrt{2} l \cos \lb \Theta + \frac{\pi}{2} \rb
	\end{aligned} \notag
\end{gather}

Положим $\Theta = \Phi = \displaystyle \frac{\pi}{2}$, тогда:
\begin{gather}
	\begin{aligned}
		X_2 &= 0 \\
		Y_2 &= l \\
		Z_2 &= 0 
	\end{aligned}
	\quad
	\begin{aligned}
		X_3 &= 0 \\
		Y_3 &= 0 \\
		Z_3 &= - \sqrt{2} l
	\end{aligned}
	\quad
	\begin{aligned}
		X_4 &= 0 \\
		Y_4 &= 0 \\
		Z_4 &= - \sqrt{2} l
	\end{aligned} \notag
\end{gather}

\end{document}


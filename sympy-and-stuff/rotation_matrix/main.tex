\documentclass[12pt]{article}

\usepackage[T1]{fontenc}
\usepackage[utf8]{inputenc}
\usepackage[russian]{babel}

% page margin
\usepackage[top=2cm, bottom=2cm, left=2cm, right=2cm]{geometry}

% AMS packages
\usepackage{amsmath}
\usepackage{amssymb}
\usepackage{amsfonts}
\usepackage{amsthm}

% blackboard lettering
\usepackage{dsfont}
\usepackage{bbm}

\usepackage{fancyhdr}
\pagestyle{fancy}

\newcommand{\bbS}{\mathbb{S}}
\newcommand{\bfj}{\mathbf{j}}
\newcommand{\bfJ}{\mathbf{J}}

\begin{document}

Ориентируем лабораторную систему координат в начальный момент времени таким образом, что вектор $\bfj$ ориентирован вдоль оси $OZ$. Матрица, связывающие координаты векторов в лабораторной и подвижной системах координат выглядит следующим образом: 
\begin{gather}
\bfj = \bbS \bfJ = 
\begin{bmatrix}
0 \\
0 \\
J
\end{bmatrix} \notag \\
\bbS = 
\begin{bmatrix} 
\cos \psi \cos \varphi - \cos \theta \sin \varphi \sin \psi & - \sin \psi \cos \varphi - \cos \theta \sin \varphi \cos \psi & \sin \theta \sin \varphi \\ \cos \psi \sin \varphi + \cos \theta \cos \varphi \sin \psi & - \sin \psi \sin \varphi + \cos \theta \cos \varphi \cos \psi & - \sin \theta \cos \varphi \\ \sin \theta \sin \psi & \sin \theta \cos \psi & \cos \theta 
\end{bmatrix} \notag
\end{gather}

Так как матрица $\bbS$ ортогональна, то $\bbS^{-1} = \bbS^\top$. Итак, получаем следующиие соотношения на углы $\theta, \psi$:
\begin{gather}
\left\{
\begin{aligned}
J_x &= J \sin \psi \sin \theta \\
J_y &= J \cos \psi \sin \theta \\
J_z &= J \cos \theta
\end{aligned}
\right. \notag
\end{gather}

Из связи между вектором угловой скорости и вектор эйлеровых скоростей получим выражение для $\dot{\varphi}$:
\begin{gather}
\mathbf{\Omega} = \mathbb{V} \dot{\mathbf{e}} = 
\begin{bmatrix}
\sin \theta \sin \psi & \cos \varphi & 0 \\
\sin \theta \cos \psi & - \sin \psi & 0 \\
\cos \theta & 0 & 1
\end{bmatrix}
\begin{bmatrix}
\dot{\varphi} \\
\dot{\theta} \\
\dot{\psi}
\end{bmatrix} 
\quad \implies \quad
\dot{\mathbf{e}} = 
\begin{bmatrix}
\displaystyle \frac{\sin \psi}{\sin \theta} & \displaystyle \frac{\cos \psi}{\sin \theta} & 0 \\
\cos \psi & - \sin \psi & 0 \\
- \sin \psi \ctg \theta & - \cos \psi \ctg \theta & 1 
\end{bmatrix}
\mathbf{\Omega} \notag \\
\dot{\varphi} = \frac{1}{\sin \theta} \left( \Omega_x \sin \psi + \Omega_y \cos \psi \right) = J \cdot \frac{J_x \Omega_x + J_y \Omega_y}{J_x^2 + J_y^2} \notag
\end{gather}

Интегрируя, получаем значение угла $\varphi(t)$:
\begin{gather}
	\varphi(t) = J \cdot \int_{0}^{t} \frac{J_x(\xi) \Omega_x(\xi) + J_y(\xi)\Omega_y(\xi)}{J_x^2(\xi) + J_y^2(\xi)} d\xi \notag
\end{gather}

Компоненты матрицы $\bbS$ несложно выразить через компоненты углового момента и $\varphi$:
\begin{gather}
	\bbS = \frac{1}{\sqrt{J_x^2 + J_y^2}} 
	\begin{bmatrix}
		J_y \cos \varphi - \displaystyle \frac{1}{J} J_x J_z \sin \varphi & J_x \cos \varphi + \displaystyle \frac{1}{J} J_y J_z \sin \varphi & \displaystyle \frac{J_x^2 + J_y^2}{J} \sin \varphi \\
		J_y \sin \varphi + \displaystyle \frac{1}{J} J_x J_z \cos \varphi & - J_x \sin \varphi + \displaystyle \frac{1}{J} J_y J_z \cos \varphi & - \displaystyle \frac{J_x^2 + J_y^2}{J} \cos \varphi \\
		\displaystyle \frac{J_x}{J} \sqrt{J_x^2 + J_y^2} & \displaystyle \frac{J_y}{J} \sqrt{J_x^2 + J_y^2} & \displaystyle \frac{J_z}{J} \sqrt{J_x^2 + J_y^2}
	\end{bmatrix} \notag
\end{gather}

\end{document}

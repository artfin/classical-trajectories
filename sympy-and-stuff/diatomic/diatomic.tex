\documentclass[12pt]{article}

\usepackage[T1]{fontenc}
\usepackage[utf8]{inputenc}
\usepackage[russian]{babel}

% page margin
\usepackage[top=2cm, bottom=2cm, left=2cm, right=2cm]{geometry}

% AMS packages
\usepackage{amsmath}
\usepackage{amssymb}
\usepackage{amsfonts}
\usepackage{amsthm}

% for strut keyword
\usepackage{mathtools}

% blackboard lettering
\usepackage{dsfont}
\usepackage{bbm}

\usepackage{fancyhdr}
\pagestyle{fancy}

\newcommand{\bbU}{\mathbb{U}}
\newcommand{\bbG}{\mathbb{G}}
\newcommand{\bbE}{\mathbb{E}}
\newcommand{\bbS}{\mathbb{S}}
\newcommand{\bbV}{\mathbb{V}}
\newcommand{\bfj}{\mathbf{j}}
\newcommand{\bfJ}{\mathbf{J}}
\newcommand{\mL}{\mathcal{L}}
\newcommand{\mH}{\mathcal{H}}

\newcommand{\mf}{\mathbf}

\newcommand{\lb}{\left(}
\newcommand{\rb}{\right)}
\newcommand{\lcb}{\left\{}
\newcommand{\rcb}{\right\}}

\begin{document}

\section{Задача двух тел и углы Эйлера}

Пусть совокупное движение двух тел происходит в плоскости $Oyz$. Выберем молекулярную систему отсчета таким образом, чтобы рассматриваемые тела находились на оси $Z$. В таком случае, плоскости молекулярной системы $OYZ$ и лабораторной системы $Oyz$ совпадают в любой момент времени. При этом угол между осями $Oy$ и $OY$ (равный, конечно, углу между $Oz$ и $OZ$) равен эйлеровому углу $\theta$ (в рамках стандартного определения эйлеровых углов по Голдстейну). Остальные два эйлеровых угла, $\phi$ и $\psi$, равны 0. Таким образом, ортогональная матрица $\bbS$, связывающая лабораторную и молекулярную систему отсчета, имеет вид:
\begin{gather}
	\bbS = \bbS_\theta = 
	\begin{bmatrix}
		1 & 0 & 0 \\
		0 & \cos \theta & \sin \theta \\
		0 & - \sin \theta & \cos \theta
	\end{bmatrix} \notag
\end{gather} 

Матрица $\bbS_\theta$, таким образом записанная, переводит координаты из лабораторной системы в молекулярную. На примере вектора углового момента:
\begin{gather}
		\mf{J} = \bbS_\theta \, \mf{j} \notag
\end{gather}

Понятно, что угловая скорость, соответсвующая повороту на угол $\theta$, направлена вдоль оси вращения $x$ = $X$. Следовательно, $\Omega_x = \dot{\theta}$. Аналогичный результат можно получить, взяв матрицу $\bbV$, связывающую компоненты угловой скорости с эйлеровыми скоростями, в общем случае и подставив в нее $\phi = \psi = 0$:
\begin{gather}
	\begin{bmatrix}
	\Omega_x \\
	\Omega_y \\
	\Omega_z 
	\end{bmatrix} = \bbV 
	\begin{bmatrix}
		\dot{\varphi} \\
		\dot{\theta} \\
		\dot{\psi}
	\end{bmatrix} \notag \\
	\bbV = \begin{bmatrix}
		\sin \theta \sin \psi & \cos \psi & 0 \\
		\sin \theta \cos \psi & - \sin \psi & 0 \\
		\cos \theta & 0 & 1
	\end{bmatrix} \rightarrow
	\begin{bmatrix}
		0 & 1 & 0 \\
		\sin \theta & 0 & 0 \\
		\cos \theta & 0 & 1
	\end{bmatrix} \notag
\end{gather}

Т.к. $\dot{\varphi} = \dot{\psi} = 0$, то $\Omega_x = \dot{\theta}$, $\Omega_y = 0$, $\Omega_z = 0$.

Обратим матрицу $\bbV$ для того, чтобы найти связь между эйлеровыми импульсами и компонентами углового момента.
\begin{gather}
	\begin{bmatrix}
		J_x \\
		J_y \\
		J_z
	\end{bmatrix} = \lb \bbV^{-1} \rb^\top
	\begin{bmatrix}
		p_\varphi \\
		p_\theta \\
		p_\psi
	\end{bmatrix} \notag \\
	\begin{bmatrix}
		J_x \\
		J_y \\
		J_z 
	\end{bmatrix} = 
	\begin{bmatrix}
		0 & 1 & 0 \\
		\displaystyle \frac{1}{\sin \theta} & 0 & - \ctg \theta \\
		0 & 0 & 1
	\end{bmatrix}
	\begin{bmatrix}
		p_\varphi \\
		p_\theta \\
		p_\psi
	\end{bmatrix} \notag
\end{gather}

Эйлеровы импульсы $p_\varphi$, $p_{\psi}$, будучи связанными с производными лагранжиана по $\dot{\varphi}$ и $\dot{\psi}$, соответственно, равны 0 (т.к. эти производные отсутствуют в угловых скоростях, следовательно и в лагранжиане). Получаем однозначную связь между $J_x$ и $p_\theta$:
\begin{gather}
	J_x = p_\theta \notag
\end{gather}

Выведем кинетическую энергию в гамильтоновой форме в предложенной системе координат. Обозначим массы тел за $m_1$ и $m_2$, расстояние между телами -- за $R$. Тогда координаты тел равны
\begin{gather}
	\begin{aligned}
			x_1 &= 0 \\
			y_1 &= 0 \\
			z_1 &= - \frac{m_2}{m_1 + m_2} R
	\end{aligned}
	\quad \quad \quad
	\begin{aligned}
			x_2 &= 0 \\
			y_2 &= 0 \\
			z_2 &= \frac{m_1}{m_1 + m_2} R 
	\end{aligned} \notag
\end{gather}

Тензор инерции будет иметь лишь две ненулевые компоненты, а именно, $I_{xx} = I_{yy} = \mu R^2$, где $\mu$ -- приведенная масса системы, $\mu = \displaystyle \frac{m_1 m_2}{m_1 + m_2}$. Приведенные выше рассуждения показывают, что вектор угловой скорости направлен вдоль оси $x = X$:
\begin{gather}
	\mf{\Omega} = \begin{bmatrix}
		\Omega_x \\
		0 \\ 
		0
	\end{bmatrix} \notag
\end{gather}

Итак, кинетическая энергия в лагранжевой форме в молекулярной системе отсчета имеет следующий вид
\begin{gather}
		T_\mL = \frac{1}{2} \mu \dot{R}^2 + \frac{1}{2} \mu R^2 \Omega_x^2 \notag 
\end{gather}

Осуществляя стандартный переход к гамильтоновой форме, получаем
\begin{gather}
	T_\mH = \frac{1}{2 \mu} p_R^2 + \frac{J_x^2}{2 \mu R^2} \notag
\end{gather}

Если заменить компоненту углового момента на эйлеров импульс, то получим
\begin{gather}
	T_\mH = \frac{1}{2 \mu} p_R^2 + \frac{p_\theta^2}{2 \mu R^2} \notag
\end{gather}

Если же зафиксировать длину связи (сделав систему палочкой):
\begin{gather}
	T_\mH = \frac{p_\theta^2}{2 \mu l^2} = \frac{1}{2 I} p_\theta^2 \notag
\end{gather}

Используем теорему Донкина для связи между компонентами угловой скорости и производными по компонентам углового момента:
\begin{gather}
	\dot{\theta} = \Omega_x = \frac{\partial \mH}{\partial J_x} = \frac{J_x}{\mu R^2} \notag \\ 
	\frac{d \theta}{dt} = \frac{J}{\mu R^2} \notag
\end{gather}

\end{document}

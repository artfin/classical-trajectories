\documentclass[12pt]{article}

\usepackage[T1]{fontenc}
\usepackage[utf8]{inputenc}
\usepackage[russian]{babel}

% page margin
\usepackage[top=2cm, bottom=2cm, left=2cm, right=2cm]{geometry}

% AMS packages
\usepackage{amsmath}
\usepackage{amssymb}
\usepackage{amsfonts}
\usepackage{amsthm}

% for strut keyword
\usepackage{mathtools}

% blackboard lettering
\usepackage{dsfont}
\usepackage{bbm}

\usepackage{fancyhdr}
\pagestyle{fancy}

\newcommand{\bbU}{\mathbb{U}}
\newcommand{\bbG}{\mathbb{G}}
\newcommand{\bbE}{\mathbb{E}}
\newcommand{\bbS}{\mathbb{S}}
\newcommand{\bfj}{\mathbf{j}}
\newcommand{\bfJ}{\mathbf{J}}
\newcommand{\mH}{\mathcal{H}}

\newcommand{\lb}{\left(}
\newcommand{\rb}{\right)}
\newcommand{\lcb}{\left\{}
\newcommand{\rcb}{\right\}}

\title{Тензорный подход к выражениям для второго спектрального момента}
\date{12 мая 2017}

\begin{document}

\maketitle

Запишем вектор дипольного момента в МСК и его производную по времени в нотации Эйнштейна, $N_\alpha$ -- орты МСК:
\begin{gather}
	\mu = \mu^\alpha N_\alpha, \quad \dot{\mu} = \dot{\mu}^\alpha N_\alpha + \mu^\alpha \dot{N_\alpha} \notag
\end{gather}

Матрица $\bbS$ связывает координаты вектора в разных базисах: $N_\alpha = \bbS^\beta_\alpha n_\beta$. Производные ортов подвижной системы могут быть представлены с использованием скобки Пуссона по эйлеровым углам и импульсам:
\begin{gather}
	\dot{N}_\alpha = \lcb N_\alpha, \mH \rcb \notag
\end{gather}

Производная вектора дипольного момента преобразуется к виду:
\begin{gather}
	\dot{\mu} = \dot{\mu}^\alpha \bbS^\beta_\alpha n_\beta + \mu^\alpha \lcb \bbS^\beta_\alpha , \mH \rcb n_\beta \notag
\end{gather}

Несложно заметить, что матричный аналог первого слагаемого есть:
\begin{gather}
	\dot{\mu}^\alpha S^\beta_\alpha n_\beta = \bbS^\top 
	\begin{bmatrix}
		\dot{\mu}_X \\
		\dot{\mu}_Y \\
		\dot{\mu}_Z
	\end{bmatrix} \notag
\end{gather}

Второе слагаемое может быть представлено в следующем виде:
\begin{gather}
	\lcb \bbS_\alpha^\beta , \mH \rcb = \lb \partial_k \bbS_\alpha^\beta \rb \lb \partial^l \mH \rb J^{k}_{l}, \notag
\end{gather}
где под $\partial$ понимается следующий дифференциальный оператор, действующий в фазовом пространстве: 
\begin{gather}
	\partial = 
	\begin{bmatrix}
		\dfrac{\strut\partial}{\strut\partial \mathbf{e}} \\
		\dfrac{\strut\partial}{\strut\partial \mathbf{p}_e}
	\end{bmatrix} \notag
\end{gather}

Несложно сообразить, что тензор $J^{k}_{l}$ имеет следующее матричное представление (в виде блочной матрицы):
\begin{gather}
	J^{k}_{l} =
	\begin{bmatrix}
		0 & \bbE \\
		-\bbE & 0
	\end{bmatrix} \notag
\end{gather}

Осуществим переход к дифференциальному оператору, содержащему производные по компонентам углового момента: 
\begin{gather}
	\widetilde{\partial} = 
	\begin{bmatrix}
		\dfrac{\strut\partial}{\strut\partial \mathbf{e}} \\
		\dfrac{\strut\partial}{\partial \mathbf{J}} \\
	\end{bmatrix} = \bbU \, \partial =  
	\begin{bmatrix}
		\bbE & 0 \\
		0 & \bbG
	\end{bmatrix}
	\begin{bmatrix}
		\dfrac{\strut\partial}{\strut\partial \mathbf{e}} \\
		\dfrac{\strut\partial}{\strut\partial \mathbf{p}_e} \\
	\end{bmatrix} \notag
\end{gather}

Осуществим замену дифференциального оператора в выражении для скобки Пуассона:
\begin{gather}
	\lcb \bbS^\beta_\alpha , \mH \rcb = \lb \partial_k \bbS^\beta_\alpha \rb \lb \partial^l \mH \rb J^k_l = \lb \partial_k \bbS^\beta_\alpha \rb \lb \bbU^l_m \widetilde{\partial}^m \mH \rb J^k_l = \lb \partial_k \bbS^\beta_\alpha \rb \lb \widetilde{\partial}^m \mH \rb \widetilde{J}^k_m, \quad \widetilde{J}^k_m = \bbU^l_m J^k_l \notag 
\end{gather}

Матричное представление тензора $\widetilde{J}_m^k$ выглядит следующим образом:
\begin{gather}
	\widetilde{J}_m^k = 
	\begin{bmatrix}
		0 & \bbE \\
		-\bbE & 0
	\end{bmatrix} 
	\begin{bmatrix}
		\bbE & 0 \\
		0 & \bbG
	\end{bmatrix} = 
	\begin{bmatrix}
		0 & \bbG \\
		-\bbE & 0
	\end{bmatrix} \notag
\end{gather}

Приходим к следующему выражению для скобки Пуассона:
\begin{gather}
	\lcb \bbS_\alpha^\beta, \mH \rcb = \lb \frac{\partial \bbS_\alpha^\beta}{\partial \mathbf{e}} \rb^\top \bbG \, \frac{\partial \mH}{\partial \bfJ}, \quad \bbG = \frac{1}{\sin \theta}
\begin{bmatrix}
	\sin \psi & \cos \psi \sin \theta & -\cos \theta \sin \psi \\
	\cos \psi & \sin \psi \sin \theta & - \cos \theta \cos \psi \\
	0 & 0 & \sin \theta
\end{bmatrix} \notag
\end{gather}

Подставляя полученный результат в выражение для производной дипольного момента и переходя к матричной нотации:
\begin{gather}
	\dot{\mu} = \bbS^\top
	\begin{bmatrix}
		\dot{\mu_X} \\
		\dot{\mu_Y} \\
		\dot{\mu_Z}
	\end{bmatrix} +
        \mathbb{W}
	\begin{bmatrix}
		\mu_X \\
		\mu_Y \\
		\mu_Z
	\end{bmatrix}, 
	\quad 
	\mathbb{W} = 
	\begin{bmatrix}
		\dfrac{\strut\partial \bbS^\top}{\strut\partial \varphi} &
		\dfrac{\strut\partial \bbS^\top}{\strut\partial \theta} &
		\dfrac{\strut\partial \bbS^\top}{\strut\partial \psi}
	\end{bmatrix}
	\bbG^\top
	\begin{bmatrix}
		\dfrac{\strut\partial \mH}{\strut\partial J_x} \\
		\dfrac{\strut\partial \mH}{\strut\partial J_y} \\
		\dfrac{\strut\partial \mH}{\strut\partial J_z}
	\end{bmatrix} \notag
\end{gather}

(матрица $\mathbb{W}$ получается как результат произведения матричного вектора, обычной матрицы и обычного вектора)
\end{document}



\section{Теория спектральных моментов}
\subsection{Спектральные моменты. Эксперимент}
Для характеристики спектров столкновительно-индуцированного поглощения часто применяются величины, называемые спектральными моментами. В общем виде выражение для $n-$го спектрального момента записывается следующим образом:

\begin{equation}
\label{Mn}
M_n=\int_{-\infty}^{+\infty}\omega^nJ(\omega)d\omega
\end{equation}
Пользуясь выражением для коэффициента поглощения, выраженного через спектральную функцию (справедливое в дипольном приближении(\textbf{??}))
\[
\alpha(\omega)=\frac{(2\pi)^3N_a^2}{3\hbar}\rho_1\rho_2\omega\Big[1-exp(-\frac{hc\omega}{kT}) \Big]J(\omega)
\]
мы можем записать выражения для нулевого и первого (второго) спектральных моментов через коэффициент поглощения:
\begin{equation}
\label{eq:mom_exp}
\begin{aligned}
\gamma_0 = \frac{1}{\rho_1\rho_2}\int_0^{+\infty}\coth(\frac{hc\omega}{2kT})\alpha(\omega)\frac{d\omega}{\omega} \\
\gamma_1 = \frac{1}{\rho_1\rho_2}\int_0^{+\infty}\alpha(\omega)d\omega
\end{aligned}
\end{equation}
Подробнее о выводах этих формул в квантовом и классическом случае см. файл \textit{mom\_quan\_class.pdf}



\subsection{Спектральные моменты. Теория}
Как мы знаем, спектральная функция задается через преобразование Фурье от автокорреляционной функции диполя:
\[
J(\omega)=\int_{-\infty}^{+\infty}C(t)e^{-i\omega t}\frac{dt}{2\pi}
\]
Применяя обратное преобразование Фурье и пользуясь формулой (\ref{Mn}), мы можем записать:
\begin{equation}
\label{autocorr_expansion}
C(t)=\int_{-\infty}^{+\infty}J(\omega)e^{i\omega t}\frac{d\omega}{2\pi} = \frac{1}{2\pi}\sum_n\frac{1}{n!}(it)^n M_n
\end{equation}
Данный переход следует из разложения экспоненты в ряд Тейлора:
\[
\begin{aligned}
\int_{-\infty}^{+\infty}J(\omega)e^{i\omega t}\frac{dt}{2\pi} = \frac{1}{2\pi}\int_{-\infty}^{+\infty}J(\omega)\bigg(1+(it)\omega+\frac{1}{2}(it)^2\omega^2+\ldots \bigg)d\omega  =\\
= \frac{1}{2\pi}\sum_n\frac{1}{n!}(it)^n\int_{-\infty}^{+\infty}J(\omega)\omega^n d\omega
\end{aligned}
\]
Таким образом, знание всего набора спектральных моментов эквивалентно знанию спектрального профиля. Однако на практике моменты выше 2-го обычно не определяют.\\

\par
Из формулы (\ref{autocorr_expansion}) следует, что в теории общее выражение для $n\text{-го}$ спектрального момента записывается следующим образом:
\begin{equation}
M_n = V(2\pi c)^{-n}i^{-n}\frac{1}{4\pi \varepsilon_0}\Big\langle\vec{\mu}(0)\cdot \frac{d^n}{dt^n}\vec{\mu}(t) \Big\rangle\Big|_{t=0}
\end{equation}
Учитывая написанное в первом разделе, имеем, что в классическом пределе для спектральных моментов:
\begin{equation}
\label{eq:mom_classical}
\begin{aligned}
M_{2n} = V(2\pi c)^{-2n}\frac{1}{4\pi \varepsilon_0}\Big\langle\Big| \frac{d^n}{dt^n}\vec{\mu}(t)\Big|^2 \Big\rangle\Big|_{t=0}\\
M_{2n+1} = 0
\end{aligned}
\end{equation}
В случае второго момента в формуле (\ref{eq:mom_classical}) первая производная может быть представлена через скобку Пуассона с классической функцией Гамильтона для рассматриваемой системы
\[
\frac{d\vec{\mu}}{dt}=[\vec{\mu},H]= \sum_j\Big\{\frac{\partial\vec{\mu}}{\partial q_j}\frac{\partial H}{\partial p_j} - \frac{\partial\vec{\mu}}{\partial p_j}\frac{\partial H}{\partial q_j}\Big\}
\]
Таким образом, выражение для второго спектрального момента будет записано следующим образом:
\begin{equation}
\label{eq:M2}
M_2 = \frac{\int\dot{\vec{\mu}}^2 e^{-H(\mathbf{p},\mathbf{q})/kT}d\mathbf{p}d\mathbf{q}}{\int e^{-H(\mathbf{p},\mathbf{q})/kT}d\mathbf{p}d\mathbf{q}}
\end{equation}
где интегрирование происходит по всем возможным начальным состояниям системы, то есть по всему фазовому пространству.
Квадрат производной вектора дипольного момента в ЛСК может быть записан следующим образом:
\[
(\dot{\vec{\mu}}^{ЛСК})^2 = (\vec{\Pi}_q)^2 + 2\vec{\Pi}_q[\frac{\partial T}{\partial J}\times \vec{M}_q] + M_q^{+}\mathbf{I}_J M_q
\]
где $T$ - кинетическая энергия рассматриваемой системы в МСК, 
\[
\vec{M}_q \equiv\left(\begin{matrix}
\mu_X \\
\mu_Y \\
\mu_Z
\end{matrix} \right) 
\]
\[
\vec{\Pi}_q \equiv\left(\begin{matrix}
[\mu_X,T]_q \\
[\mu_Y,T]_q \\
[\mu_Z,T]_q 
\end{matrix} \right) 
\]

\par
Покажем, что это действительно так.
\subsection{Выражение для квадарата производной дипольного момента}

Запишем вектор дипольного момента в подвижной (молекулярной) системе координат (МСК):
\[
\vec{\mu} = \mu_X\vec{n}_X + \mu_Y\vec{n}_Y + \mu_Z\vec{n}_Z
\]
Его производная по времени: 
\begin{equation}
\label{eq:mu_dot}
\dot{\vec{\mu}} = \dot{\mu_X}\vec{n}_X + \mu_X\dot{\vec{n}}_X +
\dot{\mu_Y}\vec{n}_Y + \mu_Y\dot{\vec{n}}_Y +
\dot{\mu_Z}\vec{n}_Z + \mu_Z\dot{\vec{n}}_Z
\end{equation}
Используем следующие обозначения:
\[
H = H(\vec{q},\vec{p},\vec{J}),\: J_{\alpha} = J_{\alpha}(\vec{e},\vec{p}_e), \: \vec{e} = \{\phi,\theta,\psi \},\: \vec{p}_e = \{p_{\phi}, p_{\theta}, p_{\psi}  \}, \: \alpha = X, Y, Z
\]
$H$ - гамильтониан системы, 
$\vec J$ -  вектор полного момента импульса в проекции на МСК,
$\vec e$ - совокупность углов Эйлера
$\vec p_e$ - совокупность сопряженных к ним импульсов,
прописными буквами обозначаются оси МСК, строчными - оси ЛСК\\

Производная по времени от некоторого свойства системы может быть выражена через скобку Пуассона:
\[
\dot{\vec{\mu}} = [\mu_X,T]_q\vec{n}_X + \mu_X[\vec{n}_X,T]_e +
[\mu_Y,T]_q\vec{n}_Y + \mu_Y[\vec{n}_Y,T]_e +
[\mu_Z,T]_q\vec{n}_Z + \mu_Z[\vec{n}_Z,T]_e
\]
Индексы $q$ и $e$ возле скобок Пуассона указывают по каким переменным берется дифференцирование (компоненты вектора $\vec\mu$ в МСК зависят только от внутренних переменных, а орты МСК зависят только от углов Эйлера). Мы имеем право писать только кинетическую энергию $T$ из функции Гамильтона, поскольку $\vec\mu$ и $\vec n$ не зависят от сопряженных импульсов, в результате чего производная функции Гамильтона будет вычисляться только по импульсам, и потенциальная энергия в выражение не войдет  \\

Переход от ЛСК к МСК
(ЛСК $\rightarrow$ МСК) осуществляется с помощью ортогональной матрицы поворота $\mathbb{S}$:

\begin{equation}
\left[\begin{matrix}
\vec{n}_X \\
\vec{n}_Y \\
\vec{n}_Z \\
\end{matrix} \right] = \mathbb{S}
 \left[\begin{matrix}
\vec{n}_x \\
\vec{n}_y \\
\vec{n}_z \\
\end{matrix} \right]
\end{equation}

Перейдем в (\ref{eq:mu_dot}) из МСК в ЛСК: 
\[
[\mu_X,T]_q\vec{n}_X = [\mu_X,T]_q(S_{11}\vec{n}_x+S_{12}\vec{n}_y+S_{13}\vec{n}_z)
\]
\[
[\mu_Y,T]_q\vec{n}_Y = [\mu_Y,T]_q(S_{21}\vec{n}_x+S_{22}\vec{n}_y+S_{23}\vec{n}_z)
\]
\[
[\mu_Z,T]_q\vec{n}_Z = [\mu_Z,T]_q(S_{31}\vec{n}_x+S_{32}\vec{n}_y+S_{33}\vec{n}_z)
\]


\[
\mu_X[\vec{n}_X,T]_e = \mu_X([S_{11},T]_e\vec{n}_x + [S_{12},T]_e\vec{n}_y +[S_{13},T]_e\vec{n}_z)
\]
\[
\mu_Y[\vec{n}_Y,T]_e = \mu_Y([S_{21},T]_e\vec{n}_x + [S_{22},T]_e\vec{n}_y +[S_{23},T]_e\vec{n}_z)
\]
\[
\mu_Z[\vec{n}_Z,T]_e = \mu_Z([S_{31},T]_e\vec{n}_x + [S_{32},T]_e\vec{n}_y +[S_{33},T]_e\vec{n}_z)
\]

Тогда производная вектора дипольного момента по времени выразится следующим образом:
\begin{equation}
\label{eq:mu_dot_expanded}
\begin{aligned}
\dot{\vec{\mu}} = \bigg \{  [\mu_X,T]_q S_{11}+  [\mu_Y,T]_q S_{21}+  [\mu_Z,T]_q S_{31} + \\
+ [S_{11},T]_e \mu_X +  [S_{21},T]_e \mu_Y +  [S_{31},T]_e \mu_Z   \bigg  \} \vec{n}_x + \\
 + \bigg\{ [\mu_X,T]_q S_{12}+  [\mu_Y,T]_q S_{22}+  [\mu_Z,T]_q S_{32} + \\
 + [S_{12},T]_e \mu_X +  [S_{22},T]_e \mu_Y +  [S_{32},T]_e \mu_Z    \bigg\} \vec{n}_y + \\
 + \bigg\{ [\mu_X,T]_q S_{13}+  [\mu_Y,T]_q S_{23}+  [\mu_Z,T]_q S_{33} + \\
 + [S_{13},T]_e \mu_X +  [S_{23},T]_e \mu_Y +  [S_{33},T]_e \mu_Z   \bigg \} \vec{n}_z 
\end{aligned}
\end{equation}

Запишем производную в краткой форме:
\[
\dot{\vec{\mu}} = \Big\{ \ldots \Big\}_x \vec{n}_x + \Big\{ \ldots \Big\}_y \vec{n}_y 
+ \Big\{ \ldots \Big\}_z \vec{n}_z
\]

Тогда её квадрат будет выражен следующим образом:
\[
\dot{\vec{\mu}}^2 = \Big\{ \ldots \Big\}^2_x + \Big\{ \ldots \Big\}^2_y 
+ \Big\{ \ldots \Big\}^2_z 
\]

Выделим в (\ref{eq:mu_dot_expanded}) слагаемые, отвечающие дифференцированию только по внутренним переменным и только по углам Эйлера. Запишем их в матричной форме:

\[
\begin{aligned}
\mathbb{S}^{+}\,\vec{\Pi}_q =  \bigg \{  [\mu_X,T]_q S_{11}+  [\mu_Y,T]_q S_{21}+  [\mu_Z,T]_q S_{31} \bigg  \} \vec{n}_x + \\
 + \bigg\{ [\mu_X,T]_q S_{12}+  [\mu_Y,T]_q S_{22}+  [\mu_Z,T]_q S_{32} \bigg\} \vec{n}_y + \\
 + \bigg\{ [\mu_X,T]_q S_{13}+  [\mu_Y,T]_q S_{23}+  [\mu_Z,T]_q S_{33}  \bigg \} \vec{n}_z 
\end{aligned}
\]

знак <<$+$>> над вектором означает транспонирование, а вектор $\vec{\Pi}_q$ имеет следующий вид:
\[
\vec{\Pi}_q \equiv\left(\begin{matrix}
[\mu_X,T]_q \\
[\mu_Y,T]_q \\
[\mu_Z,T]_q 
\end{matrix} \right) 
\]
аналогично с дифференцированием по углам Эйлера:

\begin{equation*}
\begin{aligned}
  \left[\mathbb{S}^{+},T\right]_e \vec{M}_q = \bigg \{ [S_{11},T]_e \mu_X +  [S_{21},T]_e \mu_Y +  [S_{31},T]_e \mu_Z   \bigg  \} \vec{n}_x + \\
 + \bigg\{ [S_{12},T]_e \mu_X +  [S_{22},T]_e \mu_Y +  [S_{32},T]_e \mu_Z    \bigg\} \vec{n}_y + \\
 + \bigg\{ [S_{13},T]_e \mu_X +  [S_{23},T]_e \mu_Y +  [S_{33},T]_e \mu_Z   \bigg \} \vec{n}_z  
\end{aligned}
\end{equation*}
где вектор $\vec{M}_q$ имеет следующий вид:
\[
\vec{M}_q \equiv\left(\begin{matrix}
\mu_X \\
\mu_Y \\
\mu_Z
\end{matrix} \right) 
\]

Распишем скобку Пуассона с матрицей $\mathbb{S}^{+}$:

\begin{equation}
\label{eq:poisson_S}
[\mathbb{S}^{+},T]_e  = \frac{\partial\mathbb{S}^{+}}{\partial\varphi}\frac{\partial T}{\partial p_{\varphi}} +
\frac{\partial\mathbb{S}^{+}}{\partial\theta}\frac{\partial T}{\partial p_{\theta}} +
\frac{\partial\mathbb{S}^{+}}{\partial\psi}\frac{\partial T}{\partial p_{\psi}} = 
 \mathbb{S}^{+}_{\varphi}\frac{\partial T}{\partial p_{\varphi}} +
 \mathbb{S}^{+}_{\theta}\frac{\partial T}{\partial p_{\theta}} +
 \mathbb{S}^{+}_{\psi}\frac{\partial T}{\partial p_{\psi}}
\end{equation}

\begin{equation}
\label{eq:S_e}
\begin{aligned}
\mathbb{S}^{+}_{\varphi} =  \frac{\partial\mathbb{S}^{+}}{\partial\varphi}\\
\mathbb{S}^{+}_{\theta} =  \frac{\partial\mathbb{S}^{+}}{\partial\theta}\\
 \mathbb{S}^{+}_{\psi} =  \frac{\partial\mathbb{S}^{+}}{\partial\psi}
\end{aligned}
\end{equation}

Выразим производные кинетической энергии по эйлеровым импульсам через производные по компонентам момента импульса в МСК
\[
\frac{\partial T}{\partial p_{\varphi}} = \frac{\partial T}{\partial J_X}\frac{\partial J_X}{\partial p_{\varphi}} + 
\frac{\partial T}{\partial J_Y}\frac{\partial J_Y}{\partial p_{\varphi}} + 
\frac{\partial T}{\partial J_Z}\frac{\partial J_Z}{\partial p_{\varphi}}
\]

Возвращаясь к (\ref{eq:poisson_S}), введем следующие обозначения:
\begin{equation}
\label{eq:S_phi}
\begin{aligned}
\mathbb{S}^{+}_{\varphi}\frac{\partial T}{\partial p_{\varphi}} =
\left(\mathbb{S}^{+}_{\varphi}\frac{\partial J_X}{\partial p_{\varphi}}\right) \frac{\partial T}{\partial J_X} +
\left(\mathbb{S}^{+}_{\varphi}\frac{\partial J_Y}{\partial p_{\varphi}}\right) \frac{\partial T}{\partial J_Y} +
\left(\mathbb{S}^{+}_{\varphi}\frac{\partial J_Z}{\partial p_{\varphi}}\right) \frac{\partial T}{\partial J_Z}  = \\
 = \mathbb{S}^{+}_{\varphi X} \frac{\partial T}{\partial J_X} +
\mathbb{S}^{+}_{\varphi Y} \frac{\partial T}{\partial J_Y} +
\mathbb{S}^{+}_{\varphi Z} \frac{\partial T}{\partial J_Z}
\end{aligned}
\end{equation}

Аналогично с производными по другим эйлеровым импульсам:
\begin{equation}
\mathbb{S}^{+}_{\theta}\frac{\partial T}{\partial p_{\varphi}} =
\mathbb{S}^{+}_{\theta X} \frac{\partial T}{\partial J_X} +
\mathbb{S}^{+}_{\theta Y} \frac{\partial T}{\partial J_Y} +
\mathbb{S}^{+}_{\theta Z} \frac{\partial T}{\partial J_Z}
\end{equation}

\begin{equation}
\label{eq:S_psi}
\mathbb{S}^{+}_{\psi}\frac{\partial T}{\partial p_{\varphi}} =
\mathbb{S}^{+}_{\psi X} \frac{\partial T}{\partial J_X} +
\mathbb{S}^{+}_{\psi Y} \frac{\partial T}{\partial J_Y} +
\mathbb{S}^{+}_{\psi Z} \frac{\partial T}{\partial J_Z}
\end{equation}

Компоненты момента импульса в МСК связаны с эйлеровыми импульсами следующей матрицей:
\[
\left(\begin{matrix}
J_X \\
J_Y \\
J_Z 
\end{matrix}\right)=
 \left[ \begin {array}{ccc} {\frac {\sin \left( \psi \right) }{\sin
 \left( \theta \right) }}&\cos \left( \psi \right) &-{\frac {\cos
 \left( \theta \right) \sin \left( \psi \right) }{\sin \left( \theta
 \right) }}\\ \noalign{\medskip}{\frac {\cos \left( \psi \right) }{
\sin \left( \theta \right) }}&-\sin \left( \psi \right) &-{\frac {\cos
 \left( \theta \right) \cos \left( \psi \right) }{\sin \left( \theta
 \right) }}\\ \noalign{\medskip}0&0&1\end {array} \right] 
 \left(\begin{matrix}
p_{\varphi}\\
 p_{\theta}\\
  p_{\psi} 
\end{matrix}\right)
\]

В результате члены $\mathbb{S}_{e\, \alpha}$ примут следующий вид:
\begin{equation}
\label{eq:S_ealpha}
\begin {array}{ccc} \mathbb{S}^{+}_{\varphi X} = \frac{\sin\psi}{\sin\theta} \mathbb{S}^{+}_{\varphi} &\mathbb{S}^{+}_{\varphi Y} = \frac{\cos\psi}{\sin\theta} \mathbb{S}^{+}_{\varphi} & 
\mathbb{S}^{+}_{\varphi Z} = \mathbf{0} \\
 \noalign{\medskip}\mathbb{S}^{+}_{\theta X} = \cos\psi \:\mathbb{S}^{+}_{\theta} &
 \mathbb{S}^{+}_{\theta Y} = -\sin\psi\: \mathbb{S}^{+}_{\theta} &
 \mathbb{S}^{+}_{\theta Z} = \mathbf{0} \\
  \noalign{\medskip}\mathbb{S}^{+}_{\psi X} = -\sin\psi\cot\theta \:\mathbb{S}^{+}_{\psi} &
 \mathbb{S}^{+}_{\psi Y} =-\cos\psi\cot\theta\: \mathbb{S}^{+}_{\psi} &
 \mathbb{S}^{+}_{\psi Z} = \mathbb{S}^{+}_{\psi}
 \end {array}
\end{equation}

Возвращаясь к (\ref{eq:poisson_S}) и принимая во внимание соотношения (\ref{eq:S_phi})-(\ref{eq:S_psi}), получаем:
\begin{equation}
\label{eq:W}
\begin{aligned}
\left[\mathbb{S}^{+},T\right]_e = \Big( \mathbb{S}^{+}_{\varphi X} + \mathbb{S}^{+}_{\theta X}+ \mathbb{S}^{+}_{\psi X}\Big) \frac{\partial T}{\partial J_X} + \\
+ \Big( \mathbb{S}^{+}_{\varphi Y} + \mathbb{S}^{+}_{\theta Y}+ \mathbb{S}^{+}_{\psi Y}\Big) \frac{\partial T}{\partial J_Y} + \\
+ \Big( \mathbb{S}^{+}_{\varphi Z} + \mathbb{S}^{+}_{\theta Z}+ \mathbb{S}^{+}_{\psi Z}\Big) \frac{\partial T}{\partial J_Z} = \\
\mathbb{S}^{+}_{X} \frac{\partial T}{\partial J_X} + \mathbb{S}^{+}_{Y} \frac{\partial T}{\partial J_Y} +
\mathbb{S}^{+}_{Z} \frac{\partial T}{\partial J_Z} = \mathbb{W}
\end{aligned}
\end{equation}
В результате всех преобразований производная вектора дипольного момента выражается следующим образом: 
\[
\dot{\vec{\mu}} = \mathbb{S}^{+}\vec{\Pi}_q + \mathbb{W} \vec{M}_q
\]
Квадрат вектора дпольного момента:

\begin{equation}
\begin{aligned}
\dot{\vec\mu}^2 = \dot{\vec{\mu}}^{+}\dot{\vec{\mu}} =  \big(\mathbb{S}^{+}\vec{\Pi}_q + \mathbb{W} \vec{M}_q\big)^{+}\big(\mathbb{S}^{+}\vec{\Pi}_q + \mathbb{W} \vec{M}_q\big) = \\
= \big(\vec{\Pi}_q^{+}\mathbb{S} +  \vec{M}_q^{+}\mathbb{W}^{+}\big)^{+}\big(\mathbb{S}^{+}\vec{\Pi}_q + \mathbb{W} \vec{M}_q\big) =\\
= \vec{\Pi}_q^{+}\mathbb{S}\mathbb{S}^{+}\vec{\Pi}_q + \vec{\Pi}_q^{+}\mathbb{S} \mathbb{W} \vec{M}_q +
 \vec{M}_q^{+}\mathbb{W}^{+}\mathbb{S}^{+}\vec{\Pi}_q +  \vec{M}_q^{+}\mathbb{W}^{+} \mathbb{W} \vec{M}_q
\end{aligned}
\end{equation}



\begin{center}
\line(1,0){350}
\end{center}

Подводя итоги:
Квадрат производной вектора дипольного момента от времени:

\begin{equation}
\label{eq:dip_squared}
\begin{aligned}
\dot{\vec\mu}^2 =  \vec{\Pi}_q^{+}\mathbb{S}\mathbb{S}^{+}\vec{\Pi}_q + \vec{\Pi}_q^{+}\mathbb{S} \mathbb{W} \vec{M}_q +
 \vec{M}_q^{+}\mathbb{W}^{+}\mathbb{S}^{+}\vec{\Pi}_q +  \vec{M}_q^{+}\mathbb{W}^{+} \mathbb{W} \vec{M}_q
\end{aligned}
\end{equation}

\[
\vec{\Pi}_q \equiv\left(\begin{matrix}
[\mu_X,T]_q \\
[\mu_Y,T]_q \\
[\mu_Z,T]_q 
\end{matrix} \right) 
\]

\[
\vec{M}_q \equiv\left(\begin{matrix}
\mu_X \\
\mu_Y \\
\mu_Z
\end{matrix} \right) 
\]
Произведения матриц, входящие в выражение (\ref{eq:dip_squared}) расписываются следующим образом:
\begin{equation}
\begin{aligned}
\mathbb{S}\mathbb{S}^{+} = 1\\
\mathbb{S} \mathbb{W}  = \mathbb{S}\mathbb{S}^{+}_X  \frac{\partial T}{\partial J_X} +
\mathbb{S}\mathbb{S}^{+}_Y  \frac{\partial T}{\partial J_Y} +
\mathbb{S}\mathbb{S}^{+}_Z  \frac{\partial T}{\partial J_Z} \\
\mathbb{W}^{+}\mathbb{S}^{+} = \mathbb{S}_X\mathbb{S}^{+}  \frac{\partial T}{\partial J_X} +
\mathbb{S}_Y\mathbb{S}^{+}  \frac{\partial T}{\partial J_Y} +
\mathbb{S}_Z\mathbb{S}^{+}  \frac{\partial T}{\partial J_Z} \\
\mathbb{W}^{+} \mathbb{W} = \mathbb{S}_X\mathbb{S}^{+}_X  \left(\frac{\partial T}{\partial J_X}\right)^2 +
\mathbb{S}_Y\mathbb{S}^{+}_Y  \left(\frac{\partial T}{\partial J_Y}\right)^2 +
\mathbb{S}_Z\mathbb{S}^{+}_Z  \left(\frac{\partial T}{\partial J_Z}\right)^2 +\\
+ \big(\mathbb{S}_X\mathbb{S}^{+}_Y +\mathbb{S}_Y\mathbb{S}^{+}_X    \big)\frac{\partial T}{\partial J_X}\frac{\partial T}{\partial J_Y} +\\
+ \big(\mathbb{S}_X\mathbb{S}^{+}_Z +\mathbb{S}_Z\mathbb{S}^{+}_X    \big)\frac{\partial T}{\partial J_X}\frac{\partial T}{\partial J_Z} +\\
\big(\mathbb{S}_Y\mathbb{S}^{+}_Z +\mathbb{S}_Z\mathbb{S}^{+}_Y    \big)\frac{\partial T}{\partial J_Y}\frac{\partial T}{\partial J_Z} 
\end{aligned}
\end{equation}

где согласно уравнению (\ref{eq:W}):
\[
\mathbb{S}^{+}_{X} = \mathbb{S}^{+}_{\varphi X} + \mathbb{S}^{+}_{\theta X}+ \mathbb{S}^{+}_{\psi X}
\]
\[
\mathbb{S}^{+}_{Y} = \mathbb{S}^{+}_{\varphi Y} + \mathbb{S}^{+}_{\theta Y}+ \mathbb{S}^{+}_{\psi Y}
\]
\[
\mathbb{S}^{+}_{Z} = \mathbb{S}^{+}_{\varphi Z} + \mathbb{S}^{+}_{\theta Z}+ \mathbb{S}^{+}_{\psi Z}
\]

где члены вида $\mathbb{S}_{e\, \alpha}^{+}$ определены в (\ref{eq:S_ealpha}), а члены вида $\mathbb{S}_{e}^{+}$ -- в (\ref{eq:S_e}):

\subsection{Вывод выражений для квадарата производной дипольного момента с использованием тензорной нотации}


Запишем вектор дипольного момента в МСК и его производную по времени в нотации Эйнштейна, $N_\alpha$ -- орты МСК:
\begin{gather}
	\mu = \mu^\alpha N_\alpha, \quad \dot{\mu} = \dot{\mu}^\alpha N_\alpha + \mu^\alpha \dot{N_\alpha} \notag
\end{gather}

Матрица $\bbS$ связывает координаты вектора в разных базисах: $N_\alpha = \bbS^\beta_\alpha n_\beta$. Производные ортов подвижной системы могут быть представлены с использованием скобки Пуссона по эйлеровым углам и импульсам:
\begin{gather}
	\dot{N}_\alpha = \lcb N_\alpha, \mH \rcb \notag
\end{gather}

Производная вектора дипольного момента преобразуется к виду:
\begin{gather}
	\dot{\mu} = \dot{\mu}^\alpha \bbS^\beta_\alpha n_\beta + \mu^\alpha \lcb \bbS^\beta_\alpha , \mH \rcb n_\beta \notag
\end{gather}

Несложно заметить, что матричный аналог первого слагаемого есть:
\begin{gather}
	\dot{\mu}^\alpha S^\beta_\alpha n_\beta = \bbS^\top 
	\begin{bmatrix}
		\dot{\mu}_X \\
		\dot{\mu}_Y \\
		\dot{\mu}_Z
	\end{bmatrix} \notag
\end{gather}

Второе слагаемое может быть представлено в следующем виде:
\begin{gather}
	\lcb \bbS_\alpha^\beta , \mH \rcb = \lb \partial_k \bbS_\alpha^\beta \rb \lb \partial^l \mH \rb J^{k}_{l}, \notag
\end{gather}
где под $\partial$ понимается следующий дифференциальный оператор, действующий в фазовом пространстве: 
\begin{gather}
	\partial = 
	\begin{bmatrix}
		\dfrac{\strut\partial}{\strut\partial \mathbf{e}} \\
		\dfrac{\strut\partial}{\strut\partial \mathbf{p}_e}
	\end{bmatrix} \notag
\end{gather}

Несложно сообразить, что тензор $J^{k}_{l}$ имеет следующее матричное представление (в виде блочной матрицы):
\begin{gather}
	J^{k}_{l} =
	\begin{bmatrix}
		0 & \bbE \\
		-\bbE & 0
	\end{bmatrix} \notag
\end{gather}

Осуществим переход к дифференциальному оператору, содержащему производные по компонентам углового момента: 
\begin{gather}
	\widetilde{\partial} = 
	\begin{bmatrix}
		\dfrac{\strut\partial}{\strut\partial \mathbf{e}} \\
		\dfrac{\strut\partial}{\partial \mathbf{J}} \\
	\end{bmatrix} = \bbU \, \partial =  
	\begin{bmatrix}
		\bbE & 0 \\
		0 & \bbG
	\end{bmatrix}
	\begin{bmatrix}
		\dfrac{\strut\partial}{\strut\partial \mathbf{e}} \\
		\dfrac{\strut\partial}{\strut\partial \mathbf{p}_e} \\
	\end{bmatrix} \notag
\end{gather}

Осуществим замену дифференциального оператора в выражении для скобки Пуассона:
\begin{gather}
	\lcb \bbS^\beta_\alpha , \mH \rcb = \lb \partial_k \bbS^\beta_\alpha \rb \lb \partial^l \mH \rb J^k_l = \lb \partial_k \bbS^\beta_\alpha \rb \lb \bbU^l_m \widetilde{\partial}^m \mH \rb J^k_l = \lb \partial_k \bbS^\beta_\alpha \rb \lb \widetilde{\partial}^m \mH \rb \widetilde{J}^k_m, \quad \widetilde{J}^k_m = \bbU^l_m J^k_l \notag 
\end{gather}

Матричное представление тензора $\widetilde{J}_m^k$ выглядит следующим образом:
\begin{gather}
	\widetilde{J}_m^k = 
	\begin{bmatrix}
		0 & \bbE \\
		-\bbE & 0
	\end{bmatrix} 
	\begin{bmatrix}
		\bbE & 0 \\
		0 & \bbG
	\end{bmatrix} = 
	\begin{bmatrix}
		0 & \bbG \\
		-\bbE & 0
	\end{bmatrix} \notag
\end{gather}

Приходим к следующему выражению для скобки Пуассона:
\begin{gather}
	\lcb \bbS_\alpha^\beta, \mH \rcb = \lb \frac{\partial \bbS_\alpha^\beta}{\partial \mathbf{e}} \rb^\top \bbG \, \frac{\partial \mH}{\partial \bfJ}, \quad \bbG = \frac{1}{\sin \theta}
\begin{bmatrix}
	\sin \psi & \cos \psi \sin \theta & -\cos \theta \sin \psi \\
	\cos \psi & \sin \psi \sin \theta & - \cos \theta \cos \psi \\
	0 & 0 & \sin \theta
\end{bmatrix} \notag
\end{gather}

Подставляя полученный результат в выражение для производной дипольного момента и переходя к матричной нотации:
\begin{gather}
	\dot{\mu} = \bbS^\top
	\begin{bmatrix}
		\dot{\mu_X} \\
		\dot{\mu_Y} \\
		\dot{\mu_Z}
	\end{bmatrix} +
        \mathbb{W}
	\begin{bmatrix}
		\mu_X \\
		\mu_Y \\
		\mu_Z
	\end{bmatrix}, 
	\quad 
	\mathbb{W} = 
	\begin{bmatrix}
		\dfrac{\strut\partial \bbS^\top}{\strut\partial \varphi} &
		\dfrac{\strut\partial \bbS^\top}{\strut\partial \theta} &
		\dfrac{\strut\partial \bbS^\top}{\strut\partial \psi}
	\end{bmatrix}
	\bbG^\top
	\begin{bmatrix}
		\dfrac{\strut\partial \mH}{\strut\partial J_x} \\
		\dfrac{\strut\partial \mH}{\strut\partial J_y} \\
		\dfrac{\strut\partial \mH}{\strut\partial J_z}
	\end{bmatrix} \notag
\end{gather}

(матрица $\mathbb{W}$ получается как результат произведения матричного вектора, обычной матрицы и обычного вектора)




\subsection{Вывод выражений для второго момента в частном случае Ar-CO${}_2$}


Рассмтрим, как можно получить аналитические выражения для второго спектрального момента пары Ar-$\mathrm{CO_2}$.
Второй спектральный момент по определению представляет из себя следующее выражение:

\[
M_2 = \frac{\int \frac{\mathrm{d}\vec\mu}{\mathrm{d} t} e^{-{}^H/ \! {}_{kT}} \mathrm{d}q_1 \ldots \mathrm{d}q_n \mathrm{d}p_1 \ldots \mathrm{d}p_n}{\int e^{-{}^H/ \! {}_{kT}} \mathrm{d}q_1 \ldots \mathrm{d}q_n \mathrm{d}p_1 \ldots \mathrm{d}p_n}
\]
Производную дипольного момента от времени можно представить через скобку Пуассона:

\[
\frac{\mathrm{d}\vec\mu_L}{\mathrm{d} t} = [\vec\mu , H] = \left( \frac{\partial \vec\mu_{L}}{\partial R}\frac{\partial H}{\partial p_R}     + 
\frac{\partial \vec\mu_{L}}{\partial \Theta}\frac{\partial H}{\partial p_{\Theta}} +
\frac{\partial \vec\mu_{L}}{\partial \theta}\frac{\partial H}{\partial p_{\theta}} +
\frac{\partial \vec\mu_{L}}{\partial \psi}\frac{\partial H}{\partial p_{\psi}}  +
\frac{\partial \vec\mu_{L}}{\partial \phi}\frac{\partial H}{\partial p_{\phi}} \right)
\]
где $\vec\mu_L$ обозначает вектор дипольного момента в лабораторной системе координат. Нам удобнее работать в молекулярной системе координат, поэтому для перевода вектора из молекулярной в лабораторную систему координат мы воспользуемся матрицей углов Эйлера.


\[
\mathbb{S} = \left( \begin{matrix} \cos \psi \cos \phi - \cos \theta \sin \psi \sin \phi &&  \cos \psi \sin \phi + \cos \theta \sin \psi \cos \phi && \sin \theta \sin \psi \\
- \sin \psi \cos \phi - \cos \theta \cos \psi \sin \phi && - \sin \psi \sin \phi + \cos \theta \cos \psi \cos \phi && -\sin \theta \cos \phi \\
\sin \theta \sin \psi && \sin \theta \cos \psi && \cos \theta    \end{matrix} \right)
\]

Вектор дипольного момента в лабораторной системе выражается через вектор в молекулярной системе следующим образом:
\[
\mu_{L} = \mathbb{S}^{-1} \mu_{M}
\]

Если в явном виде расписать предыдущее равенство, учитывая, что в силу симметрии системы $\mu_Y$ отсутствует, то получим следующие выражения:

\[
\mu_x = (\cos \psi \cos \phi - \cos \theta \sin \psi \sin \phi)\mu_X + \sin \theta \sin \phi \mu_Z
\]
\[
\mu_y = (\cos \psi \sin \phi + \cos \theta \sin \psi \cos \phi)\mu_X - \sin \theta \cos \phi \mu_Z
\]
\[
\mu_z =( \sin\theta\sin\psi)\mu_X + \cos\theta \mu_Z
\]

Эйлеровы импульсы связаны с проекциями момента импульса в молекулярной системе координат при помощи следующей матрицы:
\begin{equation}
\label{eq:pJ}
\vec p_{Eu} = \left( \begin{matrix} \sin\theta \sin\psi && \cos\psi && 0 \\
 \sin\theta \cos\psi && -\sin\psi && 0 \\
 \cos\theta  && 0 && 1
  \end{matrix} \right) \vec J
\end{equation}

Представим вектор дипольного момента через проекции в лабораторной системе координат:

\[
\mu_L = \vec i\mu_x + \vec j\mu_y + \vec k\mu_z
\]
а также его производную по времени:

\[
\frac{\mathrm{d}\vec\mu}{\mathrm{d} t} =  \vec i \left( \frac{\partial \mu_{x}}{\partial R}\frac{\partial H}{\partial p_R} + \ldots \right) +
 \vec j \left( \frac{\partial \mu_{y}}{\partial R}\frac{\partial H}{\partial p_R} + \ldots \right) +
  \vec k \left( \frac{\partial \mu_{z}}{\partial R}\frac{\partial H}{\partial p_R} + \ldots \right)
\]

квадрат производной по времени примет следующий вид:

\begin{equation}
\label{eq:squared}
\left(\frac{\mathrm{d}\vec\mu}{\mathrm{d} t}\right)^2 =   \left( \frac{\partial \mu_{x}}{\partial R}\frac{\partial H}{\partial p_R} + \ldots \right)^2 +
  \left( \frac{\partial \mu_{y}}{\partial R}\frac{\partial H}{\partial p_R} + \ldots \right)^2 +
   \left( \frac{\partial \mu_{z}}{\partial R}\frac{\partial H}{\partial p_R} + \ldots \right)^2
\end{equation}


Внутри каждой скобки содержатся производные Гамильтониана по всем импульсам. Перейдем от производных по Эйлеровым импульсам к производным по компонентам углового момента:

\[
\frac{\partial H}{\partial p_{\theta}} = 
\frac{\partial H}{\partial J_X}\frac{\partial J_X}{\partial p_{\theta}} +
\frac{\partial H}{\partial J_Y}\frac{\partial J_Y}{\partial p_{\theta}} + 
\frac{\partial H}{\partial J_Z}\frac{\partial J_Z}{\partial p_{\theta}}
\]

Теперь если мы вычислим производные углового момента по эйлеровым импульсам исходя из формулы (\ref{eq:pJ}), то получим следующие выражения:

\begin{equation}
\label{eq:ham_diff}
\begin{aligned}
\frac{\partial H}{\partial p_{\theta}} = 
\frac{\partial H}{\partial J_X}\cos\psi +
\frac{\partial H}{\partial J_Y}(-\sin\psi) \\
\frac{\partial H}{\partial p_{\psi}} = 
\frac{\partial H}{\partial J_X}\left(\frac{-\cos\theta\sin\psi}{\sin\theta}\right) -
\frac{\partial H}{\partial J_Y}\left(\frac{\cos\theta\cos\psi}{\sin\theta}\right) + 
\frac{\partial H}{\partial J_Z} \\
\frac{\partial H}{\partial p_{\phi}} = 
\frac{\partial H}{\partial J_X}\frac{\sin\psi}{\sin\theta} -
\frac{\partial H}{\partial J_Y}\frac{\cos\psi}{\sin\theta} 
\end{aligned}
\end{equation}

Нетрудно проверить, что кинетическая энергия для Ar-$\mathrm{CO_2}$ может быть представлен в следующем виде
\[
H = kT (x_1^2+x_2^2+x_3^2+x_4^2+x_5^2)
\]

\begin{equation}
\label{eq:system}
\begin{cases}
x_1=\frac{p_R}{\sqrt{2\mu_2 kT}} \\
x_2=\frac{p_{\Theta}}{\sqrt{2\mu_1 l^2 kT}} \\
x_3=\frac{p_{\Theta}-J_Y}{\sqrt{2\mu_2 R^2 kT}} \\
x_4=\frac{J_X+J_Z \cot\Theta }{\sqrt{2\mu_2 R^2 kT}}\\
x_5=\frac{J_Z}{\sqrt{2\mu_1 l^2 \sin^2\Theta kT}}
\end{cases}
\begin{cases}
\frac{\partial H}{\partial p_R}=\frac{2x_1}{\sqrt{2\mu_2 kT}} \\
\frac{\partial H}{\partial p_{\Theta}}=\frac{2x_2}{\sqrt{2\mu_1 l^2 kT}} + \frac{2x_3}{\sqrt{2\mu_2 R^2 kT}} \\
\frac{\partial H}{\partial J_X}=\frac{2x_4}{\sqrt{2\mu_2 R^2 kT}} \\
\frac{\partial H}{\partial J_Y}=\frac{-2x_3}{\sqrt{2\mu_2 R^2 kT}}\\
\frac{\partial H}{\partial J_Z}=\frac{2x_5}{\sqrt{2\mu_1 l^2 \sin^2\Theta kT}}+\frac{2x_4\cot\Theta}{\sqrt{2\mu_2 R^2 kT}}
\end{cases}
\end{equation}

Для краткости обозначим:
\[
\begin{cases}
A=\sqrt{2\mu_2 kT} \\
B={\sqrt{2\mu_2 R^2 kT}} \\
C= {\sqrt{2\mu_1 l^2 kT}} \\
D={\sqrt{2\mu_1 l^2 \sin^2\Theta kT}}
\end{cases}
\]

Якобиан перехода от $\{p_R,p_{\Theta},J_X, J_Y, J_Z  \}$ к $\{x_1, x_2, x_3, x_4, x_5 \}$ $[Jac]=A\cdot B^2 \cdot C \cdot D$

Числитель выражения для дипольного момента таким образом оказывается следующим:

\begin{equation}
\label{eq:mom_sphere}
M_2 =  A\cdot B^2 \cdot C \cdot D {\int \frac{\mathrm{d}\vec\mu}{\mathrm{d} t} e^{-(x_1^2+x_2^2+x_3^2+x_4^2+x_5^2)} \mathrm{d}x_1 \mathrm{d}x_2 \mathrm{d}x_3 \mathrm{d}x_4 \mathrm{d}x_5  \sin\theta \mathrm{d}\theta \mathrm{d}\psi \mathrm{d}\phi}
\end{equation}

Подставив в (\ref{eq:squared}) выражения для дипольного момента в ЛСК через его компоненты в МСК и углы Эйлера, а также (\ref{eq:ham_diff}), получаем следующее выражение:
\begin{equation*}
\begin{aligned}
\left(\frac{\mathrm{d}\vec\mu}{\mathrm{d} t}\right)^2=\left( \frac{\partial \mu_Z}{\partial R}  \right)^2\left( \frac{\partial H}{\partial p_R}  \right)^2 +
\mu_Z^2\left( \frac{\partial H}{\partial J_Y}  \right)^2 +
\left( \frac{\partial \mu_X}{\partial R}  \right)^2\left( \frac{\partial H}{\partial p_R}  \right)^2 +
\mu_X^2\left( \frac{\partial H}{\partial J_Z}  \right)^2 + \\
\mu_X^2\left( \frac{\partial H}{\partial J_Y}  \right)^2 + 
 \left( \frac{\partial \mu_Z}{\partial \Theta}  \right)^2\left( \frac{\partial H}{\partial p_{\Theta}}  \right)^2 +
\left( \frac{\partial \mu_X}{\partial \Theta}  \right)^2\left( \frac{\partial H}{\partial p_{\Theta}}  \right)^2 +
\mu_Z^2\left( \frac{\partial H}{\partial J_X}  \right)^2 - \\
- 2\left( \frac{\partial \mu_Z}{\partial R}  \right)\mu_X\frac{\partial H}{\partial J_Y}\frac{\partial H}{\partial p_R}  -
2\left( \frac{\partial \mu_Z}{\partial \Theta}  \right)\mu_X\frac{\partial H}{\partial J_Y}\frac{\partial H}{\partial p_R} - \\
- 2\left( \frac{\partial \mu_Z}{\partial \Theta}  \right)\mu_X\frac{\partial H}{\partial J_Y}\frac{\partial H}{\partial p_{\Theta}} -
2\mu_X\mu_Z\frac{\partial H}{\partial J_X}\frac{\partial H}{\partial J_Z} +
2\left( \frac{\partial \mu_X}{\partial R}  \right)\mu_Z\frac{\partial H}{\partial J_Y}\frac{\partial H}{\partial p_R} +  \\
+ 2\left( \frac{\partial \mu_X}{\partial \Theta}  \right)\mu_X\frac{\partial H}{\partial J_Y}\frac{\partial H}{\partial p_{\Theta}} +
2\left( \frac{\partial \mu_Z}{\partial R}  \right)\frac{\partial\mu_Z}{\partial \Theta}\frac{\partial H}{\partial p_{\Theta}}\frac{\partial H}{\partial p_R} + \\
+ 2\left( \frac{\partial \mu_X}{\partial R}  \right)\frac{\partial\mu_X}{\partial \Theta}\frac{\partial H}{\partial p_{\Theta}}\frac{\partial H}{\partial p_R}
\end{aligned}
\end{equation*}
Интересно отметить, что в него не вошел ни один из углов Эйлера, они взаимно сократились в ходе преобразований\\
Теперь подставляем уже полученные производные гамильтониана (\ref{eq:system}). Получаем сумму нескольких членов, из которых мы выпишем только два для демонстрации дальнейших преобразований:

\[
\frac{4\mu_Z^2 x^2_4}{B^2}-\frac{8\frac{\partial \mu_X}{R}\mu_Z x_3x_1}{BA}
\]
Подставляем в (\ref{eq:mom_sphere}), опуская якобиан для наглядности:

\begin{equation*}
M_2 =   {\int \left( \frac{4\mu_Z^2 x^2_4}{B^2}-\frac{8\frac{\partial \mu_X}{R}\mu_Z x_3x_1}{BA} \right) e^{-(x_1^2+x_2^2+x_3^2+x_4^2+x_5^2)} \mathrm{d}x_1 \mathrm{d}x_2 \mathrm{d}x_3 \mathrm{d}x_4 \mathrm{d}x_5  \sin\theta \mathrm{d}\theta \mathrm{d}\psi \mathrm{d}\phi}
\end{equation*}
Переходим к повторным интегралам:

\begin{equation*}
\begin{aligned}
M_2 = - \frac{8\frac{\partial \mu_X}{\partial R}}{BA}\mu_Z \int x_1e^{-x_1^2}  \mathrm{d}x_1 \int e^{-x_2^2}  \mathrm{d}x_2 \int x_3e^{-x_3^2}  \mathrm{d}x_3 \int e^{-x_4^2}  \mathrm{d}x_4  \int e^{-x_5^2}  \mathrm{d}x_5 \int \sin\theta \mathrm{d}\theta \mathrm{d}\psi \mathrm{d}\phi + \\
+ \frac{4\mu_Z^2}{B^2}\int e^{-x_1^2}  \mathrm{d}x_1 \int e^{-x_2^2}  \mathrm{d}x_2 \int e^{-x_3^2}  \mathrm{d}x_3 \int x_4^2 e^{-x_4^2}  \mathrm{d}x_4  \int e^{-x_5^2}  \mathrm{d}x_5 \int \sin\theta \mathrm{d}\theta \mathrm{d}\psi \mathrm{d}\phi
 \end{aligned}
\end{equation*}

Для вычисления такого рода интегралов пользуемся известной формулой:

\[
\int^{\infty}_0 x^n e^{-ax^2} \mathrm{d}x = \frac{\Gamma\left( \frac{n+1}{2}\right)}{2a^{\frac{n+1}{2}}}
\]


В результате приходим к следующему конечному выражению:

\begin{equation*}
\begin{aligned}
\mathbf{M_2} = 4\,{\frac {A{B}^{2}C{\pi }^{5/2} \left( \mu_{{X}} \left( R,\Theta
 \right)  \right) ^{2}}{ \left( D \right) {\beta}^{7/2}{{\rm e}^{\beta
\,V \left( R,\Theta \right) }}}}+4\,{\frac {A{B}^{2} \left( D \right) 
{\pi }^{5/2} \left( {\frac {\partial }{\partial \Theta}}\mu_{{X}}
 \left( R,\Theta \right)  \right) ^{2}}{C{\beta}^{7/2}{{\rm e}^{\beta
\,V \left( R,\Theta \right) }}}}+ \\
+ 4\,{\frac {A{B}^{2} \left( D \right) 
{\pi }^{5/2} \left( {\frac {\partial }{\partial \Theta}}\mu_{{Z}}
 \left( R,\Theta \right)  \right) ^{2}}{C{\beta}^{7/2}{{\rm e}^{\beta
\,V \left( R,\Theta \right) }}}}  +8\,{\frac {AC \left( D \right) {\pi }
^{5/2} \left( \mu_{{Z}} \left( R,\Theta \right)  \right) ^{2}}{{\beta}
^{7/2}{{\rm e}^{\beta\,V \left( R,\Theta \right) }}}}+4\,{\frac {AC
 \left( D \right) {\pi }^{5/2} \left( {\frac {\partial }{\partial 
\Theta}}\mu_{{Z}} \left( R,\Theta \right)  \right) ^{2}}{{\beta}^{7/2}
{{\rm e}^{\beta\,V \left( R,\Theta \right) }}}}- \\
-8\,{\frac {AC \left( D
 \right) {\pi }^{5/2} \left( {\frac {\partial }{\partial \Theta}}\mu_{
{X}} \left( R,\Theta \right)  \right) \mu_{{Z}} \left( R,\Theta
 \right) }{{\beta}^{7/2}{{\rm e}^{\beta\,V \left( R,\Theta \right) }}}
}+ 8\,{\frac {AC \left( D \right) {\pi }^{5/2} \left( {\frac {\partial 
}{\partial \Theta}}\mu_{{Z}} \left( R,\Theta \right)  \right) \mu_{{X}
} \left( R,\Theta \right) }{{\beta}^{7/2}{{\rm e}^{\beta\,V \left( R,
\Theta \right) }}}} + \\
+ 4\,{\frac {AC \left( D \right) {\pi }^{5/2}
 \left( \mu_{{X}} \left( R,\Theta \right)  \right) ^{2}}{{\beta}^{7/2}
{{\rm e}^{\beta\,V \left( R,\Theta \right) }}}} + 4\,{\frac {AC \left( D
 \right) {\pi }^{5/2} \left( {\frac {\partial }{\partial \Theta}}\mu_{
{X}} \left( R,\Theta \right)  \right) ^{2}}{{\beta}^{7/2}{{\rm e}^{
\beta\,V \left( R,\Theta \right) }}}}- \\
- 8\,{\frac {AC \left( D \right) {
\pi }^{5/2}\mu_{{X}} \left( R,\Theta \right) \mu_{{Z}} \left( R,\Theta
 \right) \cos \left( \Theta \right) }{{\beta}^{7/2}{{\rm e}^{\beta\,V
 \left( R,\Theta \right) }}\sin \left( \Theta \right) }}+4\,{\frac {AC
 \left( D \right) {\pi }^{5/2} \left( \mu_{{X}} \left( R,\Theta
 \right)  \right) ^{2} \left( \cos \left( \Theta \right)  \right) ^{2}
}{{\beta}^{7/2}{{\rm e}^{\beta\,V \left( R,\Theta \right) }} \left( 
\sin \left( \Theta \right)  \right) ^{2}}}+ \\
+ 4\,{\frac {{B}^{2}C \left( 
D \right) {\pi }^{5/2} \left( {\frac {\partial }{\partial R}}\mu_{{Z}}
 \left( R,\Theta \right)  \right) ^{2}}{A{\beta}^{7/2}{{\rm e}^{\beta
\,V \left( R,\Theta \right) }}}}+4\,{\frac {{B}^{2}C \left( D \right) 
{\pi }^{5/2} \left( {\frac {\partial }{\partial R}}\mu_{{X}} \left( R,
\Theta \right)  \right) ^{2}}{A{\beta}^{7/2}{{\rm e}^{\beta\,V \left( 
R,\Theta \right) }}}}
\end{aligned}
\end{equation*}


\subsection{Размерности спектральных моментов}
Кроме того, интересный вопрос представляют из себя размерности спектральных моментов. Рассмотрим его на примере пары $\text{Ar-CO}_2$
\paragraph{Исходные формулы}

\begin{equation}
\label{eq:gammaM}
\gamma_n=\frac{4\pi^2}{3\hbar c}M_n
\end{equation}

В классическом случае спектральные моменты с нечетным $n$ становятся равными $0$, а четные обращаются в:

\begin{equation}
\label{eq:generalmom}
M_{2n}=(2\pi c)^{-2n}V\frac{1}{4\pi\varepsilon_0}\left.\left\langle \left| \frac{d^n}{dt^n} \vec{\mu}(t) \right|^2 \right\rangle\right|_{t=0}
\end{equation}

\paragraph{Коэффициент преобразования размерностей нулевого момента} 


Переходя от производной по времени в уравнении (\ref{eq:generalmom}) к зависимости от координат и угла, получаем:

\begin{equation}
\label{eq:zemom}
M_0=\frac{1}{4\pi\varepsilon_0}\frac{4\pi}{2}\int^{\infty}_0(\mu(R,\theta))^2e^{\frac{-V(R,\theta)}{kT}}R^2sin(\theta)\mathrm{d}R\mathrm{d} \theta
\end{equation}

Размерности величин, входящих в подинтегральное выражение уравнения (\ref{eq:zemom}):\newline

\begin{center}
\begin{tabular}{|c|}
%\text{\mu(R,\theta)}
\hline
\\
\( \text{dim}[\mu(R,\theta)]=[I \cdot T \cdot L]\)\\
\\
\( \text{dim}[R]=[ L]\)\\
\\
\( \text{dim}[\mathrm{d}R]=[L]\)\\
\\
\hline
\end{tabular}
\end{center}

Поэтому dim\([\int \ldots \mathrm{d}R\mathrm{d}\theta]=[I^2\cdot T^2\cdot L^5]\), причем dim\([I\cdot T]=[\text{заряд}]\)\newline

Также необходимо учесть, что мы производим усреднение по фазовому пространству для одной пары частиц, притом что в нашей реальной системе их значительно больше. Это необходим учесть, домножив интеграл на произведение $n_An_B$ количеств каждой из частиц нашей пары в исследуемом объеме. Это можно представить как $\rho_A\rho_B\cdot n_o^2$, где $n_0$ -- константа Лошмидта.
\par
Поскольку в подинтегральном выражении все величины выражены в атомных единицах, для перевода значения нашего нулевого спектрального момента в единицы СИ, нам необходимо домножить значение интеграла, выраженное в атомных единицах на значения соответствующих атомных единиц, выраженных в СИ.
\par
Учитывя все вышесказанное, получаем для нулевого момента, выраженного в единицах СИ:
\begin{displaymath}
\gamma_0=\rho_A\rho_B \frac{4\pi^2}{3\hbar c} \frac{1}{4\pi\varepsilon_0}\frac{4\pi}{2}a_0^5\:n_0^2\:e^2\int^{\infty}_0(\mu(R,\theta))^2e^{\frac{-V(R,\theta)}{kT}}R^2sin(\theta)\mathrm{d}R\mathrm{d} \theta
\end{displaymath}
С учетом того, что \( \frac{e^2}{4\pi\varepsilon\hbar c}=\alpha_F\) -- постоянная тонкой структуры:
\begin{equation}
\gamma_0=\rho_A\rho_B \frac{4\pi^2}{3}\: \alpha_F \: a_0^5\:n_0^2\:4\pi\frac{1}{2}\int^{\infty}_0(\mu(R,\theta))^2e^{\frac{-V(R,\theta)}{kT}}R^2sin(\theta)\mathrm{d}R\mathrm{d} \theta
\end{equation}
что совпадает с уже известной формулой\footnote{Poll, Hunt, 1976, \textit{Can. J. Phys}}

\paragraph{Коэффициент для второго спектрального момента.}

В общем виде выражение для расчета второго спектрального момента методом интегрирования фазового пространства имеет следующую форму ({см. Frommhold стр. 215}) :

\begin{equation}
\label{eq:secmom}
M_2=\frac{1}{4\pi\varepsilon_0}\frac{1}{(2\pi c)^2}\:V\:\frac{\iint {F}(R,\theta)\mathrm{d}R\mathrm{d} \theta}{\iint  {G}(R,\theta)\mathrm{d}R\mathrm{d} \theta}
\end{equation}
где $V$ -- объем, выраженный в атомных единицах.\\
Обозначим интеграл из формулы (\ref{eq:secmom}) как
\begin{displaymath}
I_0=V \:\frac{\iint {F}(R,\theta)\mathrm{d}R\mathrm{d} \theta}{\iint  {G}(R,\theta)\mathrm{d}R\mathrm{d} \theta}
\end{displaymath}

Рассмотрим поочередно числитель и знаменатель под интегралами данной формулы.
\subparagraph{Числитель.}
Он состоит из суммы нескольких членов одинаковой размерности, из которых мы выпишем для примера только один:

\begin{displaymath}
Frac_1= \frac{4\pi^{^5\!/_2}D\left(\frac{\partial }{\partial \theta}\mu_X\left(R,\theta\right)\right)^2}{O\cdot F\cdot A\cdot C\cdot e^{\beta V(R,\theta)}\beta^{^7\!/_2}}
\end{displaymath}

Коэффициенты имеют следующий вид:
 
\begin{displaymath}
A=\frac{1}{\sin\theta}\sqrt{\frac{1}{2}\left( \frac{1}{\mu_1 l^2}+\frac{\cos\theta}{\mu_2R^2}  \right)}
\end{displaymath} 
\begin{displaymath}
C=\sqrt{\frac{1}{2\mu_2R^2\left( 1-\frac{1}{\mu_2R^2}\frac{\cos^2\theta}{\frac{1}{\mu_1 l^2}+\frac{\cos\theta}{\mu_2R^2} }  \right)}}
\end{displaymath}
\begin{displaymath}
D=\sqrt{\frac{1}{2\mu_2R^2}}
\end{displaymath}
\begin{displaymath}
F=\sqrt{\frac{1}{2\mu_1l^2}}
\end{displaymath}
\begin{displaymath}
O=\sqrt{\frac{1}{2\mu_2}}
\end{displaymath}
\begin{displaymath}
\beta=\frac{1}{kT}
\end{displaymath}

Нетрудно заметить, что каждый из коэффициентов (кроме $\beta$, который имеет размерность $\frac{1}{J}$, где $J$ -- энергия)  имеет размерность $\frac{1}{\sqrt{ML^2}}$. Путем несложных преобразований находим размерность $Frac_1$:

\begin{displaymath}
\text{dim}[Frac_1]=L^4M^{^3\!/_2}Z^2J^{^7\!/_2}
\end{displaymath}
где $Z$ -- заряд.

\subparagraph{Знаменатель.} Он представляет из себя одну дробь следующего вида:

\begin{displaymath}
G(R,\theta)= \frac{2\pi^{^5\!/_2} e^{-\beta V(R,\theta)}}{O\cdot D\cdot F\cdot C\cdot A\cdot\beta^{^5\!/_2}}
\end{displaymath}


Рассчитываем размерность:

\begin{displaymath}
\text{dim}[G(R,\theta)]=L^4M^{^5\!/_2}J^{^5\!/_2}
\end{displaymath}

Таким образом общая размерность $I_0$ равна:

\begin{displaymath}
\text{dim}[I_0]=\text{dim}[V]\frac{\text{dim}[Frac_1]\;\text{dim}[\mathrm{d}R]}{\text{dim}[G(R,\theta)]\;\text{dim}[\mathrm{d}R]}=\frac{Z^2J\: L^3}{M}
\end{displaymath}

Теперь вновь возвращаемся к формуле для $M_2$ (\ref{eq:secmom}) и выписываем коэффициент для преобразования размерности второго момента, принимая во внимание формулу (\ref{eq:gammaM}) и описанные в пункте про нулевой момент рассуждения относительно количества участвующих частиц

\begin{equation}
\gamma_2=\rho_A\rho_B \frac{1}{4\pi\varepsilon_0}\frac{1}{(2\pi c)^2}\: \frac{e^2E_h\:a_0^3\:n_0^2}{m_e} \frac{4\pi^2}{3\hbar c}    \frac{\iint {F}(R,\theta)\mathrm{d}R\mathrm{d} \theta}{\iint  {G}(R,\theta)\mathrm{d}R\mathrm{d} \theta}
\end{equation}
где $m_e$ -- масса электрона, $E_h$ -- энергия Хартри.\\
Нетрудно видеть, что 
\begin{displaymath}
\text{dim}[\gamma_2]=\frac{1}{L^3}
\end{displaymath}






Рассмтрим, как можно получить аналитические выражения для второго спектрального момента пары Ar-$\mathrm{CO_2}$.
Второй спектральный момент по определению представляет из себя следующее выражение:

\[
M_2 = \frac{\int \frac{\mathrm{d}\vec\mu}{\mathrm{d} t} e^{-{}^H/ \! {}_{kT}} \mathrm{d}q_1 \ldots \mathrm{d}q_n \mathrm{d}p_1 \ldots \mathrm{d}p_n}{\int e^{-{}^H/ \! {}_{kT}} \mathrm{d}q_1 \ldots \mathrm{d}q_n \mathrm{d}p_1 \ldots \mathrm{d}p_n}
\]
Производную дипольного момента от времени можно представить через скобку Пуассона:

\[
\frac{\mathrm{d}\vec\mu_L}{\mathrm{d} t} = [\vec\mu , H] = \left( \frac{\partial \vec\mu_{L}}{\partial R}\frac{\partial H}{\partial p_R}     + 
\frac{\partial \vec\mu_{L}}{\partial \Theta}\frac{\partial H}{\partial p_{\Theta}} +
\frac{\partial \vec\mu_{L}}{\partial \theta}\frac{\partial H}{\partial p_{\theta}} +
\frac{\partial \vec\mu_{L}}{\partial \psi}\frac{\partial H}{\partial p_{\psi}}  +
\frac{\partial \vec\mu_{L}}{\partial \phi}\frac{\partial H}{\partial p_{\phi}} \right)
\]
где $\vec\mu_L$ обозначает вектор дипольного момента в лабораторной системе координат. Нам удобнее работать в молекулярной системе координат, поэтому для перевода вектора из молекулярной в лабораторную систему координат мы воспользуемся матрицей углов Эйлера.


\[
\mathbb{S} = \left( \begin{matrix} \cos \psi \cos \phi - \cos \theta \sin \psi \sin \phi &&  \cos \psi \sin \phi + \cos \theta \sin \psi \cos \phi && \sin \theta \sin \psi \\
- \sin \psi \cos \phi - \cos \theta \cos \psi \sin \phi && - \sin \psi \sin \phi + \cos \theta \cos \psi \cos \phi && -\sin \theta \cos \phi \\
\sin \theta \sin \psi && \sin \theta \cos \psi && \cos \theta    \end{matrix} \right)
\]

Вектор дипольного момента в лабораторной системе выражается через вектор в молекулярной системе следующим образом:
\[
\mu_{L} = \mathbb{S}^{-1} \mu_{M}
\]

Если в явном виде расписать предыдущее равенство, учитывая, что в силу симметрии системы $\mu_Y$ отсутствует, то получим следующие выражения:

\[
\mu_x = (\cos \psi \cos \phi - \cos \theta \sin \psi \sin \phi)\mu_X + \sin \theta \sin \phi \mu_Z
\]
\[
\mu_y = (\cos \psi \sin \phi + \cos \theta \sin \psi \cos \phi)\mu_X - \sin \theta \cos \phi \mu_Z
\]
\[
\mu_z =( \sin\theta\sin\psi)\mu_X + \cos\theta \mu_Z
\]

Эйлеровы импульсы связаны с проекциями момента импульса в молекулярной системе координат при помощи следующей матрицы:
\begin{equation}
\label{eq:pJ}
\vec p_{Eu} = \left( \begin{matrix} \sin\theta \sin\psi && \cos\psi && 0 \\
 \sin\theta \cos\psi && -\sin\psi && 0 \\
 \cos\theta  && 0 && 1
  \end{matrix} \right) \vec J
\end{equation}

Представим вектор дипольного момента через проекции в лабораторной системе координат:

\[
\mu_L = \vec i\mu_x + \vec j\mu_y + \vec k\mu_z
\]
а также его производную по времени:

\[
\frac{\mathrm{d}\vec\mu}{\mathrm{d} t} =  \vec i \left( \frac{\partial \mu_{x}}{\partial R}\frac{\partial H}{\partial p_R} + \ldots \right) +
 \vec j \left( \frac{\partial \mu_{y}}{\partial R}\frac{\partial H}{\partial p_R} + \ldots \right) +
  \vec k \left( \frac{\partial \mu_{z}}{\partial R}\frac{\partial H}{\partial p_R} + \ldots \right)
\]

квадрат производной по времени примет следующий вид:

\begin{equation}
\label{eq:squared}
\left(\frac{\mathrm{d}\vec\mu}{\mathrm{d} t}\right)^2 =   \left( \frac{\partial \mu_{x}}{\partial R}\frac{\partial H}{\partial p_R} + \ldots \right)^2 +
  \left( \frac{\partial \mu_{y}}{\partial R}\frac{\partial H}{\partial p_R} + \ldots \right)^2 +
   \left( \frac{\partial \mu_{z}}{\partial R}\frac{\partial H}{\partial p_R} + \ldots \right)^2
\end{equation}


Внутри каждой скобки содержатся производные Гамильтониана по всем импульсам. Перейдем от производных по Эйлеровым импульсам к производным по компонентам углового момента:

\[
\frac{\partial H}{\partial p_{\theta}} = 
\frac{\partial H}{\partial J_X}\frac{\partial J_X}{\partial p_{\theta}} +
\frac{\partial H}{\partial J_Y}\frac{\partial J_Y}{\partial p_{\theta}} + 
\frac{\partial H}{\partial J_Z}\frac{\partial J_Z}{\partial p_{\theta}}
\]

Теперь если мы вычислим производные углового момента по эйлеровым импульсам исходя из формулы (\ref{eq:pJ}), то получим следующие выражения:

\begin{equation}
\label{eq:ham_diff}
\begin{aligned}
\frac{\partial H}{\partial p_{\theta}} = 
\frac{\partial H}{\partial J_X}\cos\psi +
\frac{\partial H}{\partial J_Y}(-\sin\psi) \\
\frac{\partial H}{\partial p_{\psi}} = 
\frac{\partial H}{\partial J_X}\left(\frac{-\cos\theta\sin\psi}{\sin\theta}\right) -
\frac{\partial H}{\partial J_Y}\left(\frac{\cos\theta\cos\psi}{\sin\theta}\right) + 
\frac{\partial H}{\partial J_Z} \\
\frac{\partial H}{\partial p_{\phi}} = 
\frac{\partial H}{\partial J_X}\frac{\sin\psi}{\sin\theta} -
\frac{\partial H}{\partial J_Y}\frac{\cos\psi}{\sin\theta} 
\end{aligned}
\end{equation}

Нетрудно проверить, что кинетическая энергия для Ar-$\mathrm{CO_2}$ может быть представлен в следующем виде
\[
H = kT (x_1^2+x_2^2+x_3^2+x_4^2+x_5^2)
\]

\begin{equation}
\label{eq:system}
\begin{cases}
x_1=\frac{p_R}{\sqrt{2\mu_2 kT}} \\
x_2=\frac{p_{\Theta}}{\sqrt{2\mu_1 l^2 kT}} \\
x_3=\frac{p_{\Theta}-J_Y}{\sqrt{2\mu_2 R^2 kT}} \\
x_4=\frac{J_X+J_Z \cot\Theta }{\sqrt{2\mu_2 R^2 kT}}\\
x_5=\frac{J_Z}{\sqrt{2\mu_1 l^2 \sin^2\Theta kT}}
\end{cases}
\begin{cases}
\frac{\partial H}{\partial p_R}=\frac{2x_1}{\sqrt{2\mu_2 kT}} \\
\frac{\partial H}{\partial p_{\Theta}}=\frac{2x_2}{\sqrt{2\mu_1 l^2 kT}} + \frac{2x_3}{\sqrt{2\mu_2 R^2 kT}} \\
\frac{\partial H}{\partial J_X}=\frac{2x_4}{\sqrt{2\mu_2 R^2 kT}} \\
\frac{\partial H}{\partial J_Y}=\frac{-2x_3}{\sqrt{2\mu_2 R^2 kT}}\\
\frac{\partial H}{\partial J_Z}=\frac{2x_5}{\sqrt{2\mu_1 l^2 \sin^2\Theta kT}}+\frac{2x_4\cot\Theta}{\sqrt{2\mu_2 R^2 kT}}
\end{cases}
\end{equation}

Для краткости обозначим:
\[
\begin{cases}
A=\sqrt{2\mu_2 kT} \\
B={\sqrt{2\mu_2 R^2 kT}} \\
C= {\sqrt{2\mu_1 l^2 kT}} \\
D={\sqrt{2\mu_1 l^2 \sin^2\Theta kT}}
\end{cases}
\]

Якобиан перехода от $\{p_R,p_{\Theta},J_X, J_Y, J_Z  \}$ к $\{x_1, x_2, x_3, x_4, x_5 \}$ $[Jac]=A\cdot B^2 \cdot C \cdot D$

Числитель выражения для дипольного момента таким образом оказывается следующим:

\begin{equation}
\label{eq:mom_sphere}
M_2 =  A\cdot B^2 \cdot C \cdot D {\int \frac{\mathrm{d}\vec\mu}{\mathrm{d} t} e^{-(x_1^2+x_2^2+x_3^2+x_4^2+x_5^2)} \mathrm{d}x_1 \mathrm{d}x_2 \mathrm{d}x_3 \mathrm{d}x_4 \mathrm{d}x_5  \sin\theta \mathrm{d}\theta \mathrm{d}\psi \mathrm{d}\phi}
\end{equation}

Подставив в (\ref{eq:squared}) выражения для дипольного момента в ЛСК через его компоненты в МСК и углы Эйлера, а также (\ref{eq:ham_diff}), получаем следующее выражение:
\begin{equation*}
\begin{aligned}
\left(\frac{\mathrm{d}\vec\mu}{\mathrm{d} t}\right)^2=\left( \frac{\partial \mu_Z}{\partial R}  \right)^2\left( \frac{\partial H}{\partial p_R}  \right)^2 +
\mu_Z^2\left( \frac{\partial H}{\partial J_Y}  \right)^2 +
\left( \frac{\partial \mu_X}{\partial R}  \right)^2\left( \frac{\partial H}{\partial p_R}  \right)^2 +
\mu_X^2\left( \frac{\partial H}{\partial J_Z}  \right)^2 + \\
\mu_X^2\left( \frac{\partial H}{\partial J_Y}  \right)^2 + 
 \left( \frac{\partial \mu_Z}{\partial \Theta}  \right)^2\left( \frac{\partial H}{\partial p_{\Theta}}  \right)^2 +
\left( \frac{\partial \mu_X}{\partial \Theta}  \right)^2\left( \frac{\partial H}{\partial p_{\Theta}}  \right)^2 +
\mu_Z^2\left( \frac{\partial H}{\partial J_X}  \right)^2 - \\
- 2\left( \frac{\partial \mu_Z}{\partial R}  \right)\mu_X\frac{\partial H}{\partial J_Y}\frac{\partial H}{\partial p_R}  -
2\left( \frac{\partial \mu_Z}{\partial \Theta}  \right)\mu_X\frac{\partial H}{\partial J_Y}\frac{\partial H}{\partial p_R} - \\
- 2\left( \frac{\partial \mu_Z}{\partial \Theta}  \right)\mu_X\frac{\partial H}{\partial J_Y}\frac{\partial H}{\partial p_{\Theta}} -
2\mu_X\mu_Z\frac{\partial H}{\partial J_X}\frac{\partial H}{\partial J_Z} +
2\left( \frac{\partial \mu_X}{\partial R}  \right)\mu_Z\frac{\partial H}{\partial J_Y}\frac{\partial H}{\partial p_R} +  \\
+ 2\left( \frac{\partial \mu_X}{\partial \Theta}  \right)\mu_X\frac{\partial H}{\partial J_Y}\frac{\partial H}{\partial p_{\Theta}} +
2\left( \frac{\partial \mu_Z}{\partial R}  \right)\frac{\partial\mu_Z}{\partial \Theta}\frac{\partial H}{\partial p_{\Theta}}\frac{\partial H}{\partial p_R} + \\
+ 2\left( \frac{\partial \mu_X}{\partial R}  \right)\frac{\partial\mu_X}{\partial \Theta}\frac{\partial H}{\partial p_{\Theta}}\frac{\partial H}{\partial p_R}
\end{aligned}
\end{equation*}
Интересно отметить, что в него не вошел ни один из углов Эйлера, они взаимно сократились в ходе преобразований\\
Теперь подставляем уже полученные производные гамильтониана (\ref{eq:system}). Получаем сумму нескольких членов, из которых мы выпишем только два для демонстрации дальнейших преобразований:

\[
\frac{4\mu_Z^2 x^2_4}{B^2}-\frac{8\frac{\partial \mu_X}{R}\mu_Z x_3x_1}{BA}
\]
Подставляем в (\ref{eq:mom_sphere}), опуская якобиан для наглядности:

\begin{equation*}
M_2 =   {\int \left( \frac{4\mu_Z^2 x^2_4}{B^2}-\frac{8\frac{\partial \mu_X}{R}\mu_Z x_3x_1}{BA} \right) e^{-(x_1^2+x_2^2+x_3^2+x_4^2+x_5^2)} \mathrm{d}x_1 \mathrm{d}x_2 \mathrm{d}x_3 \mathrm{d}x_4 \mathrm{d}x_5  \sin\theta \mathrm{d}\theta \mathrm{d}\psi \mathrm{d}\phi}
\end{equation*}
Переходим к повторным интегралам:

\begin{equation*}
\begin{aligned}
M_2 = - \frac{8\frac{\partial \mu_X}{\partial R}}{BA}\mu_Z \int x_1e^{-x_1^2}  \mathrm{d}x_1 \int e^{-x_2^2}  \mathrm{d}x_2 \int x_3e^{-x_3^2}  \mathrm{d}x_3 \int e^{-x_4^2}  \mathrm{d}x_4  \int e^{-x_5^2}  \mathrm{d}x_5 \int \sin\theta \mathrm{d}\theta \mathrm{d}\psi \mathrm{d}\phi + \\
+ \frac{4\mu_Z^2}{B^2}\int e^{-x_1^2}  \mathrm{d}x_1 \int e^{-x_2^2}  \mathrm{d}x_2 \int e^{-x_3^2}  \mathrm{d}x_3 \int x_4^2 e^{-x_4^2}  \mathrm{d}x_4  \int e^{-x_5^2}  \mathrm{d}x_5 \int \sin\theta \mathrm{d}\theta \mathrm{d}\psi \mathrm{d}\phi
 \end{aligned}
\end{equation*}

Для вычисления такого рода интегралов пользуемся известной формулой:

\[
\int^{\infty}_0 x^n e^{-ax^2} \mathrm{d}x = \frac{\Gamma\left( \frac{n+1}{2}\right)}{2a^{\frac{n+1}{2}}}
\]


В результате приходим к следующему конечному выражению:

\begin{equation*}
\begin{aligned}
\mathbf{M_2} = 4\,{\frac {A{B}^{2}C{\pi }^{5/2} \left( \mu_{{X}} \left( R,\Theta
 \right)  \right) ^{2}}{ \left( D \right) {\beta}^{7/2}{{\rm e}^{\beta
\,V \left( R,\Theta \right) }}}}+4\,{\frac {A{B}^{2} \left( D \right) 
{\pi }^{5/2} \left( {\frac {\partial }{\partial \Theta}}\mu_{{X}}
 \left( R,\Theta \right)  \right) ^{2}}{C{\beta}^{7/2}{{\rm e}^{\beta
\,V \left( R,\Theta \right) }}}}+ \\
+ 4\,{\frac {A{B}^{2} \left( D \right) 
{\pi }^{5/2} \left( {\frac {\partial }{\partial \Theta}}\mu_{{Z}}
 \left( R,\Theta \right)  \right) ^{2}}{C{\beta}^{7/2}{{\rm e}^{\beta
\,V \left( R,\Theta \right) }}}}  +8\,{\frac {AC \left( D \right) {\pi }
^{5/2} \left( \mu_{{Z}} \left( R,\Theta \right)  \right) ^{2}}{{\beta}
^{7/2}{{\rm e}^{\beta\,V \left( R,\Theta \right) }}}}+4\,{\frac {AC
 \left( D \right) {\pi }^{5/2} \left( {\frac {\partial }{\partial 
\Theta}}\mu_{{Z}} \left( R,\Theta \right)  \right) ^{2}}{{\beta}^{7/2}
{{\rm e}^{\beta\,V \left( R,\Theta \right) }}}}- \\
-8\,{\frac {AC \left( D
 \right) {\pi }^{5/2} \left( {\frac {\partial }{\partial \Theta}}\mu_{
{X}} \left( R,\Theta \right)  \right) \mu_{{Z}} \left( R,\Theta
 \right) }{{\beta}^{7/2}{{\rm e}^{\beta\,V \left( R,\Theta \right) }}}
}+ 8\,{\frac {AC \left( D \right) {\pi }^{5/2} \left( {\frac {\partial 
}{\partial \Theta}}\mu_{{Z}} \left( R,\Theta \right)  \right) \mu_{{X}
} \left( R,\Theta \right) }{{\beta}^{7/2}{{\rm e}^{\beta\,V \left( R,
\Theta \right) }}}} + \\
+ 4\,{\frac {AC \left( D \right) {\pi }^{5/2}
 \left( \mu_{{X}} \left( R,\Theta \right)  \right) ^{2}}{{\beta}^{7/2}
{{\rm e}^{\beta\,V \left( R,\Theta \right) }}}} + 4\,{\frac {AC \left( D
 \right) {\pi }^{5/2} \left( {\frac {\partial }{\partial \Theta}}\mu_{
{X}} \left( R,\Theta \right)  \right) ^{2}}{{\beta}^{7/2}{{\rm e}^{
\beta\,V \left( R,\Theta \right) }}}}- \\
- 8\,{\frac {AC \left( D \right) {
\pi }^{5/2}\mu_{{X}} \left( R,\Theta \right) \mu_{{Z}} \left( R,\Theta
 \right) \cos \left( \Theta \right) }{{\beta}^{7/2}{{\rm e}^{\beta\,V
 \left( R,\Theta \right) }}\sin \left( \Theta \right) }}+4\,{\frac {AC
 \left( D \right) {\pi }^{5/2} \left( \mu_{{X}} \left( R,\Theta
 \right)  \right) ^{2} \left( \cos \left( \Theta \right)  \right) ^{2}
}{{\beta}^{7/2}{{\rm e}^{\beta\,V \left( R,\Theta \right) }} \left( 
\sin \left( \Theta \right)  \right) ^{2}}}+ \\
+ 4\,{\frac {{B}^{2}C \left( 
D \right) {\pi }^{5/2} \left( {\frac {\partial }{\partial R}}\mu_{{Z}}
 \left( R,\Theta \right)  \right) ^{2}}{A{\beta}^{7/2}{{\rm e}^{\beta
\,V \left( R,\Theta \right) }}}}+4\,{\frac {{B}^{2}C \left( D \right) 
{\pi }^{5/2} \left( {\frac {\partial }{\partial R}}\mu_{{X}} \left( R,
\Theta \right)  \right) ^{2}}{A{\beta}^{7/2}{{\rm e}^{\beta\,V \left( 
R,\Theta \right) }}}}
\end{aligned}
\end{equation*}

